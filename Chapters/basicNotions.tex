\section{Lecture 1: January 20, 2023}

    \subsection{Definition of an Ordinary Differential Equation}

        Consider the following definitions.
        \begin{definition}{\Stop\,\,Ordinary Differential Equations}{odes}

            An ordinary differential equation is an equation of the form
            \begin{equation*}
                F(x,y(x),y'(x),\ldots,y^{(n)}(x))=0
            \end{equation*}
            where \(x\) is an independent variable and \(y\) is \(n\)th order differentiable.
            
        \end{definition}
        \vphantom
        \\
        \\
        We remark that every ordinary differential equation is valid as an expression only when we specify the values of \(x\) for which it is defined.
        \begin{definition}{\Stop\,\,Order of a Differential Equation}{orderde}

            The order of an ordinary differential equation is the highest nontrivial derivative present in the equation.
            
        \end{definition}
        \vphantom
        \\
        \\
        Consider the following ordinary differential equations. From now on, we may omit the term ``ordinary,'' as partial differential equations are not considered in this text.
        \begin{enumerate}
            \item \(F(x,y,y')=\cos(xy')+y^2y'+x^2=0\) is a first order differential equation.
            \item \(F(x,y,y',y'')=-\frac{1}{1-x^2}+y''=0\) is a second order differential equation.
            \item \(F(x,y,y',y'',y''',y'''')=e^{xy}y''''-x^2y''-\sin x=0\) is a fourth order differential equation.
        \end{enumerate}
        \pagebreak
        
\section{Lecture 2: January 23, 2023}

    \subsection{Explicit and Implicit Solutions}

        We will now consider explicit and implicit solutions to differential equations.
        \begin{definition}{\Stop\,\,Explicit Solutions to Ordinary Differential Equations}{expsolode}
            
            Let \(F(\cdots)=0\) be a differential equation defined on the interval \(I\). Then, an explicit solution to \(F\) is a function
            \begin{equation*}
                y:\mathbb{R}\to\mathbb{C}
            \end{equation*}
            for which \(y(x)\) is well-defined on some set \(X\) such that \(I\cap X\neq\emptyset\) and \(y\) satisfies the differential equation for all \(x \in I\).

        \end{definition}
        \begin{definition}{\Stop\,\,Implicit Functions}{impfunc}

            Let \(f:\mathbb{R}^2\to\mathbb{R}\) satisfying \(f(x,y)=0\). Then, \(f\) defines \(y\) as an implicit function of \(x\) if and only if
            \begin{enumerate}
                \item There exists an explicit function \(g(x)\) such that \(y=g(x)\) on some interval \(I\subseteq \mathbb{R}\).
                \item For all \(x\in I\), \(f(x,g(x))=0\).
            \end{enumerate}
            
        \end{definition}
        \begin{definition}{\Stop\,\,Implicit Solutions to Ordinary Differential Equations}{impsolode}
            
            Let \(F(\cdots)=0\) be a differential equation defined on the interval \(I\). Then, an implicit solution to \(F\) is a relation \(f(x,y)=0\) if and only if \(f\) defines \(y\) as an implicit function of \(x\) on \(I\), and if \(y(x)\) is an explicit solution to the differential equation.

        \end{definition}
        \pagebreak
        \vphantom
        \\
        \\
        Consider the following examples.
        \begin{example}{\Difficulty\,\Difficulty\,\,Verifying Explicit Solutions 1}{verexpsol1}
        
            Show that \(y(x)\) is an explicit solution to the differential equation
            \begin{equation*}
                F(x,y,y',y'')=x^2y''+xy'=0.
            \end{equation*}
            Provide the domain of definition for \(y(x)\) along with the solution set. Let \(y(x)=\log x\), where \(\log x=\log_e x\).
            \begin{proof}
                We first take the derivative of \(y(x)\) to obtain \(y'(x)=\frac{1}{x}\). The second derivative of \(y(x)\) is \(y''(x)=-\frac{1}{x^2}\). Then, we have
                \begin{align*}
                    x^2y''(x)+xy'(x)&=x^2\cdot-\frac{1}{x^2}+x\cdot\frac{1}{x} \\
                    &=-1+1=0,\quad x\neq0,
                \end{align*}
                as desired. The domain of definition for \(y(x)\) is \(\{x:x\in \mathbb{R}:x>0\}\). The differential equation is defined on \(\{x:x\in\mathbb{R}\}\). Additionally, we have the restriction \(x\neq 0\). Therefore, the solution set is 
                \begin{equation*}
                    \{x:x\in\mathbb{R}\}\cap\{x:x\in\mathbb{R}:x>0\}\cap\{x:x\in\mathbb{R}:x\neq0\}=\{x:x\in\mathbb{R}:x>0\}.
                \end{equation*}
                We can then state that \(y(x)=\log x\) is an explicit solution for \(F\) on \(\{x:x\in\mathbb{R}:x>0\}\).
            \end{proof}

        \end{example}
        \begin{example}{\Difficulty\,\Difficulty\,\,Verifying Explicit Solutions 2}{verexpsol2}
        
            Show that \(y(x)\) is an explicit solution to the differential equation
            \begin{equation*}
                F(x,y,y')=yy'-4=0.
            \end{equation*}
            Provide the domain of definition for \(y(x)\) along with the solution set. Let \(y(x)=2\sqrt{2x}\).
            \begin{proof}
                We first take the derivative of \(y(x)\) to obtain \(y'(x)=\frac{2}{\sqrt{2x}}\). Then, we have
                \begin{align*}
                    y(x)y'(x)-4&=2\sqrt{2x}\cdot\frac{2}{\sqrt{2x}}-4 \\
                    &=4-4=0,\quad x\neq0,
                \end{align*}
                as desired. The domain of definition for \(y(x)\) is \(\{x:x\in\mathbb{R}:x\geq0\}\). The differential equation is defined on \(\{x:x\in\mathbb{R}\}\). Additionally, we have the restriction \(x\neq 0\). Therefore, the solution set is 
                \begin{equation*}
                    \{x:x\in\mathbb{R}\}\cap\{x:x\in\mathbb{R}:x\geq0\}\cap\{x:x\in\mathbb{R}:x\neq0\}=\{x:x\in\mathbb{R}:x>0\}.
                \end{equation*}
                We can then state that \(y(x)=2\sqrt{2x}\) is an explicit solution for \(F\) on \(\{x:x\in\mathbb{R}:x>0\}\).
            \end{proof}
            
        \end{example}
        \begin{example}{\Difficulty\,\Difficulty\,\,Verifying Implicit Solutions 1}{verimpsol1}
            
            Determine whether \(f(x,y)=x^2+y^2+4=0\) provides an implicit solution to
            \begin{equation*}
                F(x,y,y')=2x+2y''=0.
            \end{equation*}
            Provide the intervals of solution.
            \\
            \\
            First, we determine whether \(f(x,y)\) defines \(y\) as an implicit function of \(x\). Consider the functions \(g_1(x)=\sqrt{-x^2-4}\) and \(g_2(x)=-\sqrt{-x^2-4}\); these functions are defined nowhere on \(\mathbb{R}\). Thus, \(f(x,y)=0\) does not provide an implicit solution to the differential equation.

        \end{example}
        \begin{example}{\Difficulty\,\Difficulty\,\,Verifying Implicit Solutions 2}{verimpsol2}
            
            Determine whether \(f(x,y)=xy-y^2=0\) provides an implicit solution to
            \begin{equation*}
                F(x,y,y',y'')=\frac{1}{y-x^2}y''+yy'-y=0.
            \end{equation*}
            Provide the intervals of solution.
            \\
            \\
            First, we determine whether \(f(x,y)=xy-y^2=y(x-y)\) defines \(y\) as an implicit function of \(x\). Consider the functions \(g_1(x)=0\) and \(g_2(x)=x\). Both \(g_1\) and \(g_2\) are defined on \(\mathbb{R}\). Note that \(F\) has the restriction \(y-x^2\neq0\). We see that \(f(x,g_1(x))=0\) for all \(x\in\mathbb{R}\) and \(f(x,g_2(x))=x^2-x^2=0\) for all \(x\in\mathbb{R}\). Therefore, \(f\) defines \(y\) as an implicit function of \(x\). Taking \(y=g_1(x)\) gives
            \begin{equation*}
                F(x,g_1(x),g_1'(x),g_1''(x))=\frac{1}{0-x^2}(0)+(0)(0)-(0)=-\frac{1}{x^2}=0,\quad x\neq0.
            \end{equation*}
            Then, if we take \(y=g_2(x)\), we have
            \begin{equation*}
                F(x,g_2(x),g_2'(x),g_2''(x))=\frac{1}{x-x^2}y''+xy'-x=\frac{1}{x(1-x)}(0)+x(1)-x=0, \quad x\neq 0,x\neq 1.
            \end{equation*}
            Therefore, \(f(x,y)\) provides an implicit solution to the differential equation. When providing the intervals of solution, we must explicitly pick which solution, \(g_1\) or \(g_2\), to provide the interval with respect to. For \(g_1(x)\), this is 
            \begin{equation*}
                \{x:x\in\mathbb{R}:x\neq0\}.
            \end{equation*}
            and for \(g_2(x)\), it is
            \begin{equation*}
                \{x:x\in\mathbb{R}:x\neq0,x\neq 1\}.
            \end{equation*}
            Note that we did not need to consider \(y=g_2(x)\) to show that \(f(x,y)\) provides an implicit solution.
        \end{example}
        \vphantom
        \\
        \\
        Note that it is bad practice to immediately differentiate the relation \(f(x,y)\). For example, in Example \ref{exa:verimpsol1}, if we immediately differentiated \(f\), we would indeed obtain a symbolic equivalent to the differential equation, but we would not account for the domain restrictions.

\pagebreak

\section{Lecture 3: January 25, 2023}

    \subsection{General and Particular Solutions}
    
        Consider the following definitions.
        \begin{definition}{\Stop\,\,\(n\)-Parameter Families of Solutions}{nparamfam}

            A differential equation \(F(x,y,y',\ldots,y^{(n)})=0\) possesses an \(n\)-parameter family of solutions \(y(x,c_1,\ldots,c_n)\) if and only if \(y\) is a solution for any choice of \(c_1,\ldots,c_n\in\mathbb{F}\).
            
        \end{definition}
        \begin{definition}{\Stop\,\,Particular Solutions of Differential Equations}{parsoldiffeq}

            Let \(y(x,c_1,\ldots,c_n)\) be an \(n\)-parameter family of solutions to \(F(x,y,y',\ldots,y^{(n)})=0\). Then, for each choice of \(c_1,\ldots,c_n\), we obtain one particular solution.
            
        \end{definition}
        \vphantom
        \\
        \\
        Consider the following example.
        \begin{example}{\Difficulty\,\Difficulty\,\,Finding an \(n\)-Parameter Family of Solutions}{findingnparamsol}

            Consider \(F(x,y,y',y'')=y''=0\). Note that \(F\) has solutions \(y(x)=x\) and \(y(x)=\pi\) on \(\mathbb{R}\). Both these solutions are particular, as they contain no arbitrary constants. If we take the linear combination of the solutions to obtain
            \begin{equation*}
                y(x,c_1,c_2)=c_1x+c_2,
            \end{equation*}
            we have a \(2\)-parameter family of solutions.
            
        \end{example}
        \vphantom
        \\
        \\
        Note that we will often rewrite \(y(x,c_1,\ldots,c_n)\) as \(y(x)\) even though this is an abuse of notation.
        \begin{definition}{\Stop\,\,General Solutions of Differential Equations}{gensoldiffeq}

            Let \(y(x,c_1,\ldots,c_n)\) be an \(n\)-parameter family of solutions to \(F(x,y,y',\ldots,y^{(n)})=0\). Then, \(y\) is a general solution if and only if every solution to \(F\) can be obtained from some choice of \(c_1,\ldots,c_n\).
            
        \end{definition}
        \vphantom
        \\
        \\
        In various engineering applications, the terms defined in Definition \ref{def:nparamfam} and Definition \ref{def:gensoldiffeq} are equivalent; however, this construction can break. Consider the following examples.
       \begin{enumerate}
            \item The differential equation \(F(x,y,y',y'')=y''=0\) has the general solution \(y(x,c_1,c_2)=c_1x+c_2\).
            \item The differential equation \(F(x,y,y')=y'^2+y^2=0\) has only one particular solution \(y(x)=0\).
       \end{enumerate}