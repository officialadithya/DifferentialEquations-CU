
\section{Lecture 6: February 1, 2023}

\subsection{An Introduction to Separable Differential Equations}

    We wish to solve differential equations of the form \(F(x,y,y')=f(y)\frac{\dd y}{\dd x}+g(x)=0\) where \(f\) and \(g\) are continuous on a common interval \(I\). But, what is \(\dd x\), and what is \(\dd y\)? Consider the following definition.
    \begin{definition}{\Stop\,\,Differentials}{differentials}

        Let \(y(x)\) be a differentiable function. Then, if \(\Delta x\) represents any small change in \(x\), we define
        \begin{equation*}
            \dd y(x,\Delta x)=y'(x)\Delta x
        \end{equation*}

    \end{definition}
    \vphantom
    \\
    \\
    We wish to apply Definiton \ref{def:differentials} to the function \(y(x)=x\). Consider the following theorem.
    \begin{theorem}{\Stop\,\,A Useful Lemma for a Property of Differentials}{lemmapropdiff}
        
        If \(y(x)=x\), \(\dd y(x,\Delta x)=\dd x(x,\Delta x)=\Delta x\).
        \begin{proof}
            If \(y(x)=x\), \(y'(x)=1\), so by Definition \ref{def:differentials}, \(\dd y(x,\Delta x)=\Delta x\).
        \end{proof}

    \end{theorem}
    \pagebreak
    \begin{theorem}{\Stop\,\,A Useful Property of Differentials}{propdiff}

        Let \(y(x)\) be a differentiable function. Then, \(\dd y(x,\Delta x)=y'(x)\dd x(x,\Delta x)\).
        \begin{proof}
            By Definition \ref{def:differentials}, \(\dd y(x,\Delta x)=y'(x)\Delta x\). But, by Theorem \ref{thm:lemmapropdiff}, \(\dd x(x,\Delta x)\), so
            \begin{equation*}
                \dd y(x, \Delta x)=y'(x)\Delta x=y'(x)\dd x(x,\Delta x),
            \end{equation*}
            as desired.
        \end{proof}
        More familiarly, \(\dd y=y'(x)\dd x\).
    \end{theorem}
    \vphantom
    \\
    \\
    We may use Theorem \ref{thm:propdiff} to prove a version of the familar chain rule.
    \begin{theorem}{\Stop\,\,The Chain Rule for Differentials}{chainrule}

        Let \(y=f(x)\) be a differentiable function, and \(x(t)=g(t)\). Therefore, \(y(x)=f(g(t))\). Then,
        \begin{equation*}
            \dd y(t,\Delta t)=f'(x(t))\dd x(t,\Delta t).
        \end{equation*}
        \begin{proof}
            Since \(x=g(t)\), we have \(\dd x(t, \Delta t)=g'(t)\dd t(t, \Delta t)\). Then, using the chain rule for derivatives, we obtain the chain rule for differentials below
            \begin{equation*}
                \dd y(t,\Delta t)=f'(x(t))g'(t)(t,\Delta t),
            \end{equation*}
            as desired.
        \end{proof}
        
    \end{theorem}
    \vphantom
    \\
    \\
    Note that if \(z(x,y)\) is a function differentiable with respect to both \(x\) and \(y\), we obtain
    \begin{equation*}
        \dd z(x,y,\Delta x,\Delta y)=\frac{\partial z}{\partial y}\dd y+\frac{\partial z}{\partial x}\dd x.
    \end{equation*}
    Suppose we have a first order differential equation in the form \(F(x,y,y')=f(y)\frac{\dd y}{\dd x}+g(x)=0\). We use that fact that \(\dd y\) and \(\dd x\) are differential functions to write \(F(x,y,y')=f(y)\dd y+g(x)\dd x=0\). Then, we obtain the \(1\)-parameter family of solutions
    \begin{equation*}
        \int f(y)\dd y+\int g(x)\dd x=c.
    \end{equation*}
    \pagebreak
    \vphantom
    \\
    \\
    Consider the following examples.
    \begin{example}{\Difficulty\,\Difficulty\,\,Separation of Variables 1}{sepvar1}

        Consider \(F(x,y,y')=\frac{\dd y}{\dd x}+y=0\), therefore, \(\dd y=-y\dd x\). Assuming that \(y(x)\neq0\), we can divide by \(y\) to give
        \begin{equation*}
            \frac{1}{y}\dd y=-\dd x.
        \end{equation*} 
        We have a separable differential equation, and by integration, we have
        \begin{equation*}
            \int \frac{1}{y}\dd y=\int -\dd x
        \end{equation*}
        meaning that
        \begin{equation*}
            \log|y|=-x+c_1.
        \end{equation*}
        This provides an implicit solution. Note that it is a \(1\)-parameter family of solutions, but is not a general function since \(y(x)=0\) is a particular solution to the differential equation, not obtainable from the family. We seek to find an explicit solution. Consider
        \begin{equation*}
            e^{\log|y|}=e^{-x+c_1}=e^{-x}e^{c_1}
        \end{equation*}
        Then, let \(c_2=e^{c_1}\). Note that \(c_2>0\). Then, we have \(|y|=c_2e^{-x}\). This \(1\)-parameter family can be made a general solution by allowing \(c_2=0\). Our general solution is then
        \begin{equation*}
            |y|=c_2e^{-x},\quad c_2\geq0, \quad x\in\mathbb{R}.
        \end{equation*}
        If we allow \(c_2\in\mathbb{R}\), we can also write that a general solution is
        \begin{equation*}
            y=c_2e^{-x},\quad c_2\in\mathbb{R},\quad x\in\mathbb{R}.
        \end{equation*}
        This method does not always work.
        
    \end{example}
    \pagebreak
    \begin{example}{\Difficulty\,\Difficulty\,\,Separation of Variables 2}{sepvar2}

        Consider \(F(x,y,y')=x^2(y-2)\frac{\dd y}{\dd x}-y^3=0\). Assuming that \(x\neq0\) and \(y(x)\neq0\), we can write
        \begin{equation*}
            \frac{1}{x^2}\dd x=\frac{y-2}{y^3}\dd y.
        \end{equation*}
        We have a separable differential equation, and by integration, we have
        \begin{equation*}
            \int \frac{1}{x^2}\dd x=\int \frac{y-2}{y^3}\dd y
        \end{equation*}
        meaning that
        \begin{align*}
            -\frac{1}{x}&=\int \left(\frac{1}{y^2}-\frac{2}{y^3} \right)\dd y \\
            &=-\frac{1}{y}+\frac{1}{y^2}+c_1.
        \end{align*}
        We may multiply by \(y^2\) to obtain
        \begin{equation*}
            -\frac{y^2}{x}=-y+1+c_1y^2,
        \end{equation*}
        and by \(x\) to obtain
        \begin{equation*}
            -y^2=-xy+x+c_1xy^2,
        \end{equation*}
        so
        \begin{equation*}
            y^2(c_1x+1)=x(y-1),\quad y(x)\neq 0,\quad x\neq 0
        \end{equation*}
        is a \(1\)-parameter family of solutions, but it is not general because \(y(x)=0\) is a valid particular solution of \(F\) on \(\mathbb{R}\).
    \end{example}

    \pagebreak
    \begin{example}{\Difficulty\,\Difficulty\,\,Separation of Variables 3}{sepvar3}

        Consider \(F(x,y,y')=x\sqrt{1-y}-y'\sqrt{1-x^2}=0\). Assuming that \(x^2\neq 1\) and \(y\neq 1\), we can write
        \begin{equation*}
            \frac{x}{\sqrt{1-x^2}}\dd x=\frac{1}{\sqrt{1-y}}\dd y.
        \end{equation*}
        We have a separable differential equation, and by integration, we have
        \begin{equation*}
            \int \frac{x}{\sqrt{1-x^2}}\dd x=\int \frac{1}{\sqrt{1-y}}\dd y.
        \end{equation*}
        meaning that
        \begin{align*}
            -\sqrt{1-x^2}=-2\sqrt{1-y}+c_1
        \end{align*}
        is our \(1\)-parameter family of solutions. We must restrict our solution for \(|x|<1\) and \(y<1\) due to the square roots. Our \(1\)-parameter family of solutions is not general because \(y(x)=1\) is a valid particular solution of \(F\) on the interval.
        
    \end{example}
    \begin{example}{\Difficulty\,\Difficulty\,\,Separation of Variables 4}{sepvar4}

        Consider \(F(x,y,y')=x^2y'+1=0\). In differential form, we obtain \(x^2\dd y+\dd x=0\). With the restriction \(x\neq0\), we have \(\dd y=-\frac{1}{x^2}\dd x\). By integration, we have the \(1\)-parameter family of solutions \(y(x)= \frac{1}{x}+c\) on the interval \(\mathbb{R}-\{0\}\). 

    \end{example}
    \begin{example}{\Difficulty\,\Difficulty\,\,Separation of Variables 5}{sepvar5}

        Consider \(F(x,y,y')=x^2y'+y-1=0\). In differential form, we obtain \(x^2\dd y+(y-1)\dd x=0\). With the restrictions \(x\neq0\) and \(y(x)\neq 1\), we have \(\frac{1}{y-1}\dd y=-\frac{1}{x^2}\dd x\). By integration, we have the \(1\)-parameter family of solutions \(\log|y-1|=\frac{1}{x}+c\) on the interval \(\mathbb{R}-\{0\}\).

    \end{example}
    

    \pagebreak

\section{Lecture 7: February 3, 2023}

\subsection{Differential Equations with Homogeneous Coefficients}

    Consider the following definitions.
    \begin{definition}{\Stop\,\,Homogeneous Functions}{homofunc}

        The function \(f(x,y)\) is homogeneous, of order \(n\), on some region \(B\subseteq\mathbb{R}^2\) if and only if for all \(x,y\in B\), either of the below hold.
        \begin{enumerate}
            \item The function \(f(tx,ty)=t^nf(x,y)\) for some \(n\in\mathbb{N}\).
            \item The function \(f(x,y)=x^ng(u)\) for some \(u=\frac{y}{x}\) and \(n\in\mathbb{N}\).
            \item The function \(f(x,y)=y^nh(u)\) for some \(u=\frac{x}{y}\) and \(n\in\mathbb{N}\).
        \end{enumerate}

    \end{definition}
    \begin{definition}{\Stop\,\,Differential Equations With Homogeneous Coefficients}{diffeqhomocoeff}

        A first order differential equation \(F(x,y,y')=Q(x,y)\frac{\dd y}{\dd x}+P(x,y)=0\) has homogeneous coefficients if and only if both \(P(x,y)\) and \(Q(x,y)\) are both homogeneous functions of equal order.
        
    \end{definition}
    \vphantom
    \\
    \\
    Consider the following examples.
    \begin{example}{\Difficulty\,\Difficulty\,\,Is it Homogeneous? 1}{isithomo1}

        Determine whether \(f(x,y)=3x^2y-y^3\) is homogeneous on its domain.
        \\
        \\
        Consider
        \begin{align*}
            f(tx,ty)&=3(tx)^2(ty)-(ty)^3 \\
            &=3t^3x^2y-t^3y^3 \\
            &=t^3(3x^2y-y^3) \\
            &=t^3f(x,y).
        \end{align*}
        Therefore, \(f(x,y)\) is homogeneous of order \(3\).

    \end{example}
    \begin{example}{\Difficulty\,\Difficulty\,\,Is it Homogeneous? 2}{isithomo2}

        Determine whether \(f(x,y)=xy\sin(xy)\) is homogeneous on its domain.
        \\
        \\
        Consider \(f(tx,ty)=t^2xy\sin(t^2xy)\). There is no way in which this can be reduced to satisfy Definition \ref{def:homofunc}.
    \end{example}
    \pagebreak
    \begin{example}{\Difficulty\,\Difficulty\,\,Is it Homogeneous? 3}{isithomo3}

        Determine whether \(f(x,y)=xy\sin\left(\frac{x}{y}\right)-x^2\) is homogeneous on its domain.
        \\
        \\
        Consider
        \begin{align*}
            f(tx,ty)&=(tx)(ty)\sin\left(\frac{tx}{ty}\right)-(tx)^2 \\
            &=t^2xy\sin\left(\frac{x}{y}\right)-t^2x^2 \\
            &=t^2(xy\sin\left(\frac{x}{y}\right)-x^2) \\
            &=t^2f(x,y).
        \end{align*}
        Therefore, \(f(x,y)\) is homogeneous of order \(2\).
    \end{example}
    \vphantom
    \\
    \\
    While the first condition is, usually, easiest to use to show that a function is homogeneous, the other conditions are very useful in edge cases and for proofs. Consider the following theorem.
    \begin{theorem}{\Stop\,\,Differential Equations With Homogeneous Coefficients are Separable}{homodiffsep}
        
        If \(F(x,y,y')=Q(x,y)y'+P(x,y)=0\) has homogeneous coefficients, \(F\) can be solved using separation of variables.
        \begin{proof}
            If \(F(x,y,y')\) has homogeneous coefficients, \(Q(x,y)=x^ng_1(u)\) and \(P(x,y)=x^ng_2(u)\). We may make the substitution
            \begin{equation*}
                F(x,y,y')=x^ng_1(u)\frac{\dd y}{\dd x}+x^ng_2(u)=0.
            \end{equation*}
            Then, since \(u=\frac{y}{x}\), \(y=ux\), so \(y'=u+x\frac{\dd u}{\dd x}\). Therefore,
            \begin{align*}
                F(x,y,y')&=x^ng_1(u)\left(u+x\frac{\dd u}{\dd x}\right)+x^ng_2(u)=0 \\
                &=ux^ng_1(u)+x^{n+1}g_1(u)\frac{\dd u}{\dd x}+x^ng_2(u)=0 \\
                &=ux^ng_1(u)\dd x+x^{n+1}g_1(u)\dd u+x^ng_2(u)\dd x=0 \\
                &=\frac{g_1(u)}{x}\dd x+\frac{g_1(u)}{u}\dd u+\frac{g_2(u)}{ux}\dd x=0,\quad ux^{n+1}\neq0 \\
                &=\frac{\dd x}{x}\left(g_1(u)+\frac{g_2(u)}{u}\right)+\frac{g_1(u)}{u}\dd u=0 \\
                &=\frac{1}{x}\dd x+\frac{g_1(u)}{ug_1(u)+g_2(u)}\dd u=0, \quad g_1(u)+\frac{g_2(u)}{u}\neq0,\quad u\neq0
            \end{align*}
            so \(F(x,y,y')\) is separable.
        \end{proof}

    \end{theorem}

    \pagebreak

\section{Lecture 8: February 6, 2023}

    \subsection{Using the Homogeneous Substitution to Solve Differential Equations}

        Consider the following example.
        \begin{example}{\Difficulty\,\Difficulty\,\,A Homogeneous Substitution 1}{homosub1}
            
            Find a \(1\)-parameter family of solutions for 
            \begin{equation*}
                F(x,y,y')=2xy\frac{\dd y}{\dd x}-(x^2+y^2)=0.
            \end{equation*}
            \\
            \\
            We may rewrite the above as
            \begin{equation*}
                2xy\dd y-(x^2+y^2)\dd x=0.
            \end{equation*}
            We see that \(2(tx)(ty)=t^2(2xy)\) and \(-((tx)^2+(ty)^2)=-(t^2x^2+t^2y^2)=-t^2(x^2+y^2)\) so our differential equation has homogeneous coefficients of order \(2\). Let \(y=ux\), meaning \(\frac{\dd y}{\dd x}=u+x\frac{\dd u}{\dd x}\). Then, our differential equation is
            \begin{align*}
                0&=2ux^2\left(u+x\frac{\dd u}{\dd x}\right)-(x^2+u^2x^2) \\
                &=2u^2x^2+2ux^3\frac{\dd u}{\dd x}-x^2-u^2x^2 \\
                &=x^2(u^2-1)\dd x+2ux^3\dd u \\
                &=\frac{1}{x}\dd x+\frac{2u}{u^2-1},\quad x\neq0,\quad u^2-1\neq0.
            \end{align*}
            Then, we may integrate to obtain
            \begin{equation*}
                c_1=\log|x|+\log|u^2-1|.
            \end{equation*}
            If we let \(c_2=e^{c_1}\), we further obtain
            \begin{equation*}
                c_2=|x||u^2-1|
            \end{equation*}
            By our earlier substitution, we have 
            \begin{equation*}
                c_2=|x|\left|\left(\frac{y}{x}\right)^2-1\right|.
            \end{equation*}
            Our restrictions are \(x\neq0\), \(y(x)\neq x\); note that \(y(x)=x\) is a particular solution, so our \(1\)-parameter solution is not general.

        \end{example}
        \pagebreak
        \begin{example}{\Difficulty\,\Difficulty\,\,A Homogeneous Substitution 2}{homosub2}
            
            Find a \(1\)-parameter family of solutions for 
            \begin{equation*}
                F(x,y,y')=\left(x\log\frac{y}{x}-x\right)y'+y=0.
            \end{equation*}
            \\
            \\
            We may rewrite the above as
            \begin{equation*}
                \left(x\log\frac{y}{x}-x\right)\dd y+y\dd x=0.
            \end{equation*}
            We see that \(tx\log\frac{ty}{tx}-tx=t\left(x\log\frac{y}{x}-x\right)\) and \(ty=ty\) so our differential equation has homogeneous coefficients of order \(2\). Let \(y=ux\), meaning \(\dd y=u\dd x+x\dd u\). Then, our differential equation is
            \begin{align*}
                0&=\left(x\log\left(\frac{ux}{x}\right)-x\right)(u\dd x+x\dd u)+ux\dd x \\
                &=\left(x\log\left(u\right)-x\right)(u\dd x+x\dd u)+ux\dd x \\
                &=ux\log(u)\dd x+x^2\log(u)\dd u-ux\dd x-x^2\dd u+ux\dd x \\
                &=ux\log(u)\dd x+x^2(\log(u)-1)\dd u.
            \end{align*}
            With the restrictions \(x\neq0\) and \(u\log(u)\neq0\), we have
            \begin{equation*}
                \frac{1}{x}\dd x=-\frac{\log(u)-1}{u\log(u)}\dd u=\left(-\frac{1}{u}+\frac{1}{u\log(u)}\right)\dd u.
            \end{equation*}
            By integration, we have
            \begin{equation*}
                \log|x|=-\log|u|+\log|\log|u||+c,
            \end{equation*}
            which by our previous substitution, gives
            \begin{equation*}
                \log|x|=-\log\left|\frac{y}{x}\right|+\log\left|\log\left|\frac{y}{x}\right|\right|+c,
            \end{equation*}
            on the interval \(\mathbb{R}-\{0\}\).
        \end{example}
        \vphantom
        \\
        \\
        Note that in Example \ref{exa:homosub1}, we used \(\frac{\dd y}{\dd x}=u+x\frac{\dd u}{\dd x}\), but in Example \ref{exa:homosub2}, we used \(\dd y=u\dd x+x\dd u\). The latter is much cleaner and will therefore be used onwards.

    \pagebreak

    \subsection{Exactness}

        Consider the following definitions.
        \begin{definition}{\Stop\,\,Simply Connected Regions}{simpconreg}

            A set \(B\subseteq\mathbb{R}^2\) is simply connected if and only if every non-intersecting closed curve lying in \(B\) contains only points of \(B\).
            
        \end{definition}
        \begin{definition}{\Stop\,\,Exact Differential Equations}{exactdiffeq}

            The differential equation \(P(x,y)\dd x+Q(x,y)\dd y=0\) is exact if and only if \(P(x,y)\), \(\frac{\partial P}{\partial y}\), \(Q(x,y)\), and \(\frac{\partial Q}{\partial x}\) are all continuous on a common simply connected region \(B\) and
            \begin{equation*}
                \frac{\partial P}{\partial y}=\frac{\partial Q}{\partial x}.
            \end{equation*}
        \end{definition}
        \begin{theorem}{\Stop\,\,The Existence of a Potential}{potentialexist}

            If the differential equation \(P(x,y)\dd x+Q(x,y)\dd y=0\) is exact, there exists \(f(x,y)\) such that \(\frac{\partial f}{\partial x}=P(x,y)\) and \(\frac{\partial f}{\partial y}=Q(x,y)\). The function \(f(x,y)\) can be considered as a potential function for the vector field \(\vec{F}=[P(x,y),Q(x,y)]\).
            
        \end{theorem}
        \vphantom
        \\
        \\
        As motivation, let \(z=f(x,y)\). Recall that
        \begin{equation*}
            \dd z=\frac{\partial f}{\partial x}\dd x+\frac{\partial f}{\partial y}\dd y.
        \end{equation*}
        If we take \(\frac{\partial f}{\partial x}=P(x,y)\) and \(\frac{\partial f}{\partial y}=Q(x,y)\), as stipulated in Definition \ref{def:exactdiffeq}, the right hand side of the above becomes
        \begin{equation*}
            P(x,y)\dd x+Q(x,y)\dd y.
        \end{equation*}

    \pagebreak

\section{Lecture 9: February 8, 2022}

    \subsection{Solving Exact Differential Equations}

        Recall the Fundamental Theorem of Calculus. That is, for some function \(H(x)\), we have
        \begin{equation*}
            H(x)=\int_{t_0}^t H'(t)\dd t.
        \end{equation*}
        \vphantom
        \\
        \\
        Then, if \(H(x)=x\), we have \(x=\int_{t_0}^t \dd x\). Recall that for \(z=f(x,y)\), 
        \begin{equation*}
            \dd z = \frac{\partial f}{\partial x}\dd x+\frac{\partial f}{\partial y} \dd y,
        \end{equation*}
        and if we take \(\frac{\partial f}{\partial x}=P(x,y)\) and \(\frac{\partial f}{\partial y}=Q(x,y)\),
        \begin{equation*}
            \dd z=P(x,y)\dd x+Q(x,y)\dd y.
        \end{equation*}
        Notice that \(P(x,y)\dd x+Q(x,y)\dd y=0\) is exact. Let the associated simply connected region be \(B\). From the total differential of \(z\), we have
        \begin{equation*}
            f(x,y)=\int_B \dd z=\int_B P(x,y)\dd x+\int_B Q(x,y)\dd y.
        \end{equation*}
        Note that all the above integrals exist, as by Definition \ref{def:exactdiffeq}, \(P(x,y)\) and \(Q(x,y)\) are continuous. Consider the following examples.
        \begin{example}{\Difficulty\,\Difficulty\,\,Exact Differential Equation 1}{exact1}

            Solve \(F(x,y,y')=2y\frac{\dd y}{\dd x}-\cos x\frac{\dd y}{\dd x}+y\sin x=0\).
            \\
            \\
            We must first write the differential equation in differential form to produce
            \begin{equation*}
                (2y-\cos x)\dd y+y\sin x\dd x=0.
            \end{equation*}
            Then, note that \(\frac{\partial}{\partial y}y\sin x=\sin x\) and \(\frac{\partial}{\partial x}=\sin x\). Since the partial derivatives are equal, and \(P\), \(Q\), \(\frac{\partial Q}{\partial x}\), and \(\frac{\partial P}{\partial y}\) are all continuous on the simply connected region \(\mathbb{R}^2\), the differential equation is exact. We must now find some \(z=f(x,y)\) such that \(\frac{\partial f}{\partial x}=y\sin x\) and \(\frac{\partial f}{\partial y}=2y-\cos x\). Then,
            \begin{align*}
                f(x,y)&=\int y\sin x\dd x \\
                &=-y\cos x+c(y).
            \end{align*}
            Note that, now, we have \(\frac{\partial f}{\partial y}=-\cos x+c'(y)=2y-\cos x\). Therefore, \(c'(y)=2y\), and
            \begin{align*}
                f(x,y)&=\int c'(y)\dd y-y\cos x \\
                &=y^2-y\cos x+c_1.
            \end{align*}
            Then, the \(1\)-parameter family of solutions \(f(x,y,c_1)=y^2-y\cos x+c_1=0\) defines an implicit solution to \(F(x,y,y')\) on \(\mathbb{R}\).
        \end{example}

\pagebreak

\section{Lecture 10: Februrary 10, 2022}

    \subsection{Integrating Factors}

        Consider the differential equation
        \begin{equation*}
            P(x,y)\dd x+Q(x,y)\dd y=0,
        \end{equation*}
        and let \(P(x,y)\), \(\frac{\partial P}{\partial x}\), \(Q(x,y)\), \(\frac{\partial Q}{\partial x}\) all be continuous on some common simply connected region \(B\); however, \(\frac{\partial P}{\partial y}\neq\frac{\partial Q}{\partial x}\) and there does not exist some \(z=f(x,y)\) such that \(\dd z=P(x,y)\dd x+Q(x,y)\dd y\). This motivates the following definition. We will be able to find some exact differential equation possessing the same solutions as the original differential equation.
        \begin{definition}{\Stop\,\,Integrating Factors}{intfact}

            Let \(P(x,y)\dd x+Q(x,y)\dd x=0\), and let \(P(x,y)\), \(\frac{\partial P}{\partial x}\), \(Q(x,y)\), \(\frac{\partial Q}{\partial x}\) all be continuous on some common simply connected region \(B\). The function \(\mu(x,y)\) is an integrating factor if and only if the differential equation
            \begin{equation*}
                \mu(x,y)P(x,y)\dd x+\mu(x,y)Q(x,y)\dd y=0.
            \end{equation*}
            is exact. 
        \end{definition}
        \begin{theorem}{\Stop\,\,Sameness of Solutions}{sameexactsol}

            If the integrating factor \(\mu(x,y)\neq0\) for all \((x,y)\) and has continuous first-order partial derivatives on the simply connected region \(B\). Then,
            \begin{equation*}
                P(x,y)\dd x+Q(x,y)\dd y=0
            \end{equation*}
            has the same solution as 
            \begin{equation*}
                \mu(x,y)P(x,y)\dd x+\mu(x,y)Q(x,y)\dd y=0
            \end{equation*}
            \begin{proof}
                Let \(y(x)\) be a solution to the original differential equation. Then, consider
                \begin{align*}
                    \mu(x,y(x))P(x,y(x))\dd x+\mu(x,y(x))Q(x,y(x))\dd y&=\mu(x,y(x))(P(x,y(x))\dd x+\mu(x,y)Q(x,y(x))\dd y) \\
                    &=\mu(x,y(x))(0) \\
                    &=0,
                \end{align*}
                so \(y(x)\) is also a solution to
                \begin{equation*}
                    \mu(x,y)P(x,y)\dd x+\mu(x,y)Q(x,y)\dd y=0
                \end{equation*}
                on the same intervals of solution. On the other hand, if \(\tilde{y(x)}\) is a solution to the exact differential equation, we have that
                \begin{equation*}
                    \mu(x,\tilde{y}(x))(P(x,\tilde{y}(x))\dd x+\mu(x,\tilde{y})Q(x,\tilde{y}(x))\dd y)=0,
                \end{equation*}
                so either \(\mu(x,\tilde{y}(x))=0\) or \(P(x,\tilde{y}(x))\dd x+\mu(x,\tilde{y})Q(x,\tilde{y}(x))\dd y=0\). By supposition, \(\mu(x,y)\neq0\) for all \((x,y)\), so it must be the case that
                \begin{equation*}
                    P(x,\tilde{y}(x))\dd x+\mu(x,\tilde{y})Q(x,\tilde{y}(x))\dd y=0,
                \end{equation*}
                meaning that \(\tilde{y}(x)\) is a solution to the original differential equation.
            \end{proof}
            
        \end{theorem}
        \vphantom
        \\
        \\
        We seek to find the integrating factor \(\mu(x,y)\) when it is purely a function of \(x\). That is, \(\mu(x,y)=\mu(x)\). If \(\mu(x)\) is an integrating factor, we have that
        \begin{equation*}
            \mu(x)P(x,y)\dd x+\mu(x)Q(x,y)\dd y=0,
        \end{equation*}
        where 
        \begin{equation*}
            \frac{\partial}{\partial y}(\mu(x)P(x,y))=\frac{\partial}{\partial x}(\mu(x)Q(x,y)).
        \end{equation*}
        By differentiation, we have 
        \begin{equation*}
            \mu(x)\frac{\partial P}{\partial y}=\mu(x)\frac{\partial Q}{\partial x}+Q\frac{\partial \mu}{\partial x}.
        \end{equation*}
        By rearrangement, we obtain
        \begin{equation*}
            \mu(x)\left(\frac{\partial P}{\partial y}-\frac{\partial Q}{\partial x}\right)=Q\frac{\dd \mu}{\dd x}.
        \end{equation*}
        By the suppositions \(h(x)\neq0\) and \(Q(x,y)\neq0\), we may divide and see
        \begin{equation*}
            \frac{1}{\mu(x)}\frac{\dd \mu}{\dd x}=\frac{1}{Q(x,y)}\left(\frac{\partial P}{\partial y}-\frac{\partial Q}{\partial x}\right)
        \end{equation*}
        By recognizing the left hand side as the natural logarithm's derivative, we multiply by the differential function \(\dd x\) and antidifferentiate to obtain
        \begin{equation*}
            \int \frac{1}{\mu(x)}\frac{\dd \mu}{\dd x} \dd x=\int \frac{1}{Q(x,y)}\left(\frac{\partial P}{\partial y}-\frac{\partial Q}{\partial x}\right)\dd x,
        \end{equation*} 
        which implies
        \begin{equation*}
            \log|\mu(x)|=\int \frac{1}{Q(x,y)}\left(\frac{\partial P}{\partial y}-\frac{\partial Q}{\partial x}\right)\dd x.
        \end{equation*}
        Let \(H(x)\) be the integrand of the right hand side. We may exponentiate to get
        \begin{equation*}
            |\mu(x)|=e^{\int H(x)\dd x}.
        \end{equation*}
        We need only consider \(\mu(x)=e^{\int H(x)\dd x}\), as any constants multiplied by \(\mu(x)\) will not change the above derivation significantly. For other cases, where \(\mu(x,y)\) is not purely a function of \(x\), the derivation is much more complicated. Note that in the case that \(\mu(x,y)=\mu(y)\), let
        \begin{equation*}
            \tilde{H}(x)=\frac{1}{P(x,y)}\left(\frac{\partial Q}{\partial x}-\frac{\partial P}{\partial y}\right),
        \end{equation*}
        and \(\mu(x,y)=\mu(y)=e^{\int \tilde{H}(y) \dd y}\).
        \pagebreak
        \\
        \\
        As a review of the previous solution techniques, consider the following example.
        \begin{example}{\Difficulty\,\Difficulty\,\,3 Different Solution Techniques}{3difftechs}

            Solve \(F(x,y,y')=2yy'+6x=0\) using separation of variables, homogeneous substitution, and integrating factors and exactness.
            \\
            \\
            In differential form, we have
            \begin{equation*}
                2y\dd y=-6x\dd x,
            \end{equation*}
            which integrates to give us \(y^2=-3x^2+c_1\) as our implicit solution on \(\mathbb{R}\). Before performing the homogeneous substitution, note that the equation \(2y\dd y+6x\dd x=0\) has homogeneous coefficients of order \(1\) trivially. We make the substitution \(y=ux\) to obtain
            \begin{align*}
                0&=2ux(u\dd x+x\dd u)+6x\dd x \\
                &=2u^2x\dd x+2ux^2\dd u+6x\dd x \\
                &=x(2u^2+6)\dd x+2ux^2\dd u.
            \end{align*}
            With the restriction \(x\neq0\), we may divide by \(x^2\) and \(2u^2+6\) to obtain
            \begin{equation*}
                \frac{1}{x}\dd x+\frac{u}{u^2+3}\dd u,
            \end{equation*}
            and integration gives \(\log|x|=\log|\sqrt{u^2+3|}+c_2\) meaning, if \(\ell=e^{c_2}\), \(|x|=\ell\sqrt{\frac{y^2}{x^2}+3}=\sqrt{\frac{y^2+3x^2}{x^2}}\). Then, if \(k=\ell^2\),
            \begin{equation*}
                x^2=k\left(\frac{y^2+3x^2}{x^2}\right)
            \end{equation*}
            The differential equation is also trivially exact on \(\mathbb{R}^2\).
            
        \end{example}
        \begin{example}{An Differential Equation With an Integrating Factor Not a Function of \(x\)}{intfactnotx}
            
            Show that \(2xy^3\dd x+(2y+x^2y^2)\dd y=0\) is not exact and \(\mu(x,y)\neq\mu(x)\).
            \\
            \\
            Let \(P(x,y)=2xy^3\), so \(\frac{\partial P}{\partial x}=6xy^2\). Let \(Q(x,y)=2y+x^2y^2\), so \(\frac{\partial Q}{\partial x}=2xy^2\). Therefore, the differential equation is not exact. To show that the differential equation's integrating factor is not purely a function of \(x\), suppose, for the sake of contradiction, that \(\mu(x,y)=\mu(x)\). Then,
            \begin{equation*}
                \mu(x)=e^{\int H(x)\dd x},\quad H(x)=\frac{1}{Q(x,y)}\left(\frac{\partial P}{\partial y}-\frac{\partial Q}{\partial x}\right)=\frac{4xy^2}{2y+x^2y^2}.
            \end{equation*}
            Since \(H(x)\) cannot be written as a pure function of \(x\), we have a contradiction, and \(\mu(x,y)\neq\mu(x)\). No such issue occurs if we assume 
            \begin{equation*}
                \mu(x,y)=\mu(y)=e^{\int \tilde{H}(y)\dd y}, \quad \tilde{H}(y)=\frac{1}{P(x,y)}\left(\frac{\partial Q}{\partial x}-\frac{\partial P}{\partial y}\right)=-\frac{2}{y}.
            \end{equation*}
            Note that another integrating factor is \(\hat{\mu}(x,y)=e^{x^2y}\).

        \end{example}

\pagebreak

\section{Lecture 11: February 13, 2022}

    \subsection{Linear Ordinary Differential Equations}

        Consider the following definition.
        \begin{definition}{\Stop\,\,Linear Ordinary Differential Equations}{nthorderlineardiffeq}

            An \(n\)th order linear ordinary differential equation is
            \begin{equation*}
                F(x,y,y',\ldots,y^{(n)})=f_n(x)y^{(n)}+\cdots+f_1(x)y'+f_0(x)y+Q(x)=0
            \end{equation*}
            where all \(f_i\), \(0\leq i\leq n\), are continuous on some common interval \(I\) and \(f_n(x)\neq0\) for all \(x\in I\).
            
        \end{definition}
        \vphantom
        \\
        \\
        Note that every first-order linear differential equation can be written as 
        \begin{equation*}
            F(x,y,y')=f_1(x)y'+f_0(x)y+Q(x)=y'+\tilde{P}(x)y+\tilde{Q}(x)=0.
        \end{equation*}
        and has the integrating factor \(\mu(x,y)=\mu(x)=e^{\int \tilde{P}(x)\dd x}\).