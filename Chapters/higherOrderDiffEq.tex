\section{Lecture 11: February 13, 2023}

    \subsection{Linear Ordinary Differential Equations}

        Consider the following definition.
        \begin{definition}{\Stop\,\,Linear Ordinary Differential Equations}{nthorderlineardiffeq}

            An \(n\)th order linear ordinary differential equation, with respect to \(I\), is
            \begin{equation*}
                F(x,y,y',\ldots,y^{(n)})=f_n(x)y^{(n)}+\cdots+f_1(x)y'+f_0(x)y-Q(x)=0
            \end{equation*}
            where all \(f_i\), \(0\leq i\leq n\), and \(Q\) are continuous on some common interval \(I\) and \(f_n(x)\neq0\) for all \(x\in I\).
            
        \end{definition}
        \vphantom
        \\
        \\
        Note that every first-order linear differential equation can be written as 
        \begin{equation*}
            F(x,y,y')=\tilde{P}(x)y+\tilde{Q}(x)=0.
        \end{equation*}
        and has the integrating factor \(\mu(x,y)=\mu(x)=e^{\int \tilde{P}(x)\dd x}\). After multiplying through, we use the product rule for differentials \(\dd(uv)=u\dd v+v\dd u\).
        \\
        \\
        Consider the following non-example of a linear ordinary differential equation.
        \begin{example}{\Difficulty\,\Difficulty\,\,A Non-Example of a Linear Ordinary Differential Equation}{linorddiffeq}

            Consider \(F(x,y,y',y'',y'')=y'''+y''\log(x-1)+y\arccos(x)-\log(x-2)=0\).
            \\
            \\
            The differential equation \(F(x,y,y',y'',y''')\) is not a linear ordinary differential equation since \(\log(x-1)\), \(\arccos(x)\), and \(\log(x-2)\) are not continuous on any common interval.
            
        \end{example}
        \vphantom
        \\
        \\
        We now provide an example of solving a first-order linear differential equation using integrating factors.
        \begin{example}{\Difficulty\,\Difficulty\,\,Solving a First-Order Linear Equation Using Integrating Factors}{firstlinintfact}
            
            Solve \(F(x,y,y')=y'-4y\tan x+\sin x=0\).
            \\
            \\
            We have that the integrating factor is given by
            \begin{align*}
                \mu(x)&=\exp\left(\int -4\tan x\dd x\right) \\
                &=\exp(-4\log(\sec x)) \\
                &=\sec^{-4}x.
            \end{align*}
            Then,
            \begin{equation*}
                \mu(x)F(x,y,y')=y'\sec^{-4}x-4y\tan x\sec^{-4}x+\sin x\sec^{-4}x=0. 
            \end{equation*}
            In differential form, we have
            \begin{equation*}
                \sec^{-4}x\dd y-4y\tan x\sec^{-4}x\dd x+\sin x\sec^{-4}x\dd x=0.
            \end{equation*}
            Then, we can obtain, by the product rule for differentials,
            \begin{equation*}
                \dd(y\sec^{-4}x)+\sin x\sec^{-4}x\dd x=0.
            \end{equation*}
            By integration, we obtain
            \begin{align*}
                c&=y\sec^{-4}x+\int \sin x\cos^{4}x\dd x \\
                &=y\sec^{-4}x-\frac{\cos^5x}{5}
            \end{align*}
            Then, we have 
            \begin{align*}
                y(x)&=\frac{1}{\sec^{-4}x}\left(c+\frac{\cos^5x}{5}\right) \\
                &=c\sec^4x+\frac{\cos x}{5},
            \end{align*}
            on \(\{x\in\mathbb{R}:x\neq \frac{\pi}{2}+k\pi, k\in\mathbb{Z}\}\), as desired.
        \end{example}
        \pagebreak

\section{Lecture 12: February 20, 2023}

    \subsection{Existence and Uniqueness of Solutions for Linear Differential Equations}

        Consider the following definition.
        \begin{theorem}{\Stop\,\,Linear Independence}{linindep}

            Let \(S=\{f_i:X\to\mathbb{C}:1\leq i\leq n\}\) be a set of continuous complex-valued functions defined on \(X\subseteq\mathbb{R}\). Then, \(S\) is linearly independent if and only if, for all \(x\in X\), 
            \begin{equation*}
                c_1f_1(x)+\cdots+c_nf_n(x)=0
            \end{equation*}
            with \(c_1,\ldots,c_n\in\mathbb{C}\), is satisfied only if \(c_1=\cdots=c_n=0\). Note that \(S\) is linearly dependent if and only if \(S\) is not linearly independent.
            
        \end{theorem}
        \vphantom
        \\
        \\
        Consider the following theorems.
        \begin{theorem}{\Stop\,\,Existence of Solutions to \(n\)th Order Linear Differential Equations (I)}{existence1}
            
            Let \(F(x,y,\ldots,y^{(n)})\) be an \(n\)th order linear differential equation with respect to \(I\). Then, there exists \(y(x)\) such that
            \begin{equation*}
                y(x_0)=y_0,\quad y'(x_0)=y_1,\quad \ldots,\quad y^{(n-1)}(x_0)=y_{n-1}
            \end{equation*}
            are initial conditions on \(F\) satisfied for any \(x_0\in I\).

        \end{theorem}
        \begin{theorem}{\Stop\,\,Existence of Solutions to \(n\)th Order Linear Differential Equations (II)}{existence2}
            
            Let \(F(x,y,\ldots,y^{(n)})\) be an \(n\)th order linear differential equation with respect to \(I\). Then, there exist \(n\) \(1\)-parameter families of solutions \(c_iy_i(x)\) for \(c_i\in\mathbb{C}\) and some particular solution \(y_i(x)\) valid on \(I\). Moreover, the set \(S=\{y_1(x),\ldots,y_n(x)\}\) is linearly independent and \(c_1y_1(x)+\cdots+c_ny_n(x)\) is an \(n\)-parameter family of solutions valid on \(I\).

        \end{theorem}
        \begin{theorem}{\Stop\,\,Uniqueness of Solutions to \(n\)th Order Linear Differential Equations}{uniqueness}
            
            Let \(F(x,y,\ldots,y^{(n)})\) be an \(n\)th order linear differential equation with respect to \(I\). Then, \(F\) has a unique solution valid on \(I\) satisfying \(n\) arbitrary initial conditions.

        \end{theorem}
        \vphantom
        \\
        \\
        Therefore, by Theorem \ref{thm:uniqueness}, the \(n\)-parameter family of solutions specified in Theorem \ref{thm:existence2} is the \(y(x)\) specified in Theorem \ref{thm:existence1}; note that \(y(x)\) is a general solution.
        \pagebreak
        \vphantom
        \\
        \\
        \begin{definition}{\Stop\,\,Wronskians}{wronskians}

            Let \(S=\{f_i:X\to\mathbb{C}:1\leq i\leq n\}\) be a set of continuous complex-valued functions defined on \(X\subseteq\mathbb{R}\). Suppose that each \(f_i\in S\), \(1\leq i\leq n\) us \((n-1)\)-times continuously differentiable. Then, the Wronskian of \(S\) is given by
            \begin{equation*}
                W(S)(x)=\det\begin{bmatrix}
                    f_1(x) & \cdots & f_n(x) \\
                    \vdots & \ddots & \vdots \\
                    f_1^{(n-1)}(x) & \cdots & f_n^{(n-1)}(x).
                \end{bmatrix}
            \end{equation*}
            
        \end{definition}
        \begin{theorem}{\Stop\,\,Wronskians Determine Linear Independence\(^*\)}{wronskianslinindep}

            If \(S\) is linearly dependent on \(X\), \(W(S)(x)=0\) for all \(x\in X\). If \(W(S)(x)\neq0\) for some \(x\in X\), \(S\) is linearly independent on \(X\).
            \\
            \\
            \(^*\)Wronksians are not necessary and sufficient to determine linear independence. 
            
        \end{theorem}
        \vphantom
        \\
        \\
        Consider the following example.
        \begin{example}{Showing Linear Independence With the Wronskian}{showlinindepwronksian}

            Show that \(\{x,e^{2x},e^{3x}\}\) are linearly independent on \(\{x\in\mathbb{R}:0\leq x\leq 1\}\).
            \\
            \\
            Note that all elements of \(S\) are \((n-1)\)-times continuously differentiable. Then,
            \begin{align*}
                W(S)(x)&=\det\begin{bmatrix}
                    x & e^{2x} & e^{3x} \\
                    1 & 2e^{2x} & 3e^{3x} \\
                    0 & 4e^{2x} & 9e^{3x}
                \end{bmatrix} \\
                &=-6xe^{5x}+5e^{5x}.
            \end{align*}
            Note that \(W(S)(0)=5\neq0\), so \(S\) is linearly independent.
            
        \end{example}
        \begin{theorem}{\Stop\,\,Wronksians are Necessary and Sufficient in Some Cases}{wronskianslinindep2}

            If \(S=\{y_i:X\to\mathbb{C}\}\) and all \(y_i\), \(1\leq i\leq n\), are solutions to the same \(n\)th order linear differential equation, \(W(S)(x)=0\) on \(X\) if and only if \(S\) is linearly dependent.
            
        \end{theorem}

\pagebreak

\section{Lecture 13: Feburary 22, 2023}

    \subsection{Linear Differential Equations With Constant Coefficients}

        We will now study linear differential equations with constant coefficients. Consider the following definitions.
        \begin{definition}{\Stop\,\,Linear Differential Equations With Constant Coefficients}{constcoeff}

            An \(n\)th order linear differential equation, with respect to \(I\), is 
            \begin{equation*}
                F(x,y,y',\ldots,y^{(n)})=f_n(x)y^{(n)}+\cdots+f_1(x)y'+f_0(x)y-Q(x)=0,
            \end{equation*}
            where all \(f_i=a_i\), \(0\leq i\leq n\), where \(a_i\in\mathbb{C}\) and \(Q\) is continuous on some common interval \(I\) and \(f_n(x)\neq0\) for all \(x\in I\).
            
        \end{definition}
        \begin{definition}{\Stop\,\,Homogeneous Linear Differential Equations}{homolindiffeq}

            An \(n\)th order linear homogeneous differential equation, with respect to \(I\), is 
            \begin{equation*}
                F(x,y,y',\ldots,y^{(n)})=f_n(x)y^{(n)}+\cdots+f_1(x)y'+f_0(x)y-Q(x)=0,
            \end{equation*}
            where all \(f_i\), \(0\leq i\leq n\), and \(Q\) are continuous on some common interval \(I\) and \(f_n(x)\neq0\) and \(Q(x)=0\) for all \(x\in I\). 
            
        \end{definition}
        \vphantom
        \\
        \\
        Consider the following theorem.
        \begin{theorem}{\Stop\,\,Decomposing a Solution of a Linear Differential Equation}{decompsol}

            Let \(F(x,y,y',\ldots,y^{(n)})=f_n(x)y^{(n)}+\cdots+f_1(x)y'+f_0(x)y=0\) have the solution
            \begin{equation*}
                y_c(x,c_1,\ldots,c_n)=c_1y_1(x)+\cdots+c_ny_n(x).
            \end{equation*}
            Then, the solution to
            \begin{equation*}
                \tilde{F}(x,y,y',\ldots,y^{(n)})=f_n(x)y^{(n)}+\cdots+f_1(x)y'+f_0(x)-Q(x)=0
            \end{equation*}
            is given by
            \begin{equation*}
                y(x,c_1,\ldots,c_n)=y_c(x,c_1,\ldots,c_n)+y_p(x)
            \end{equation*}
            where \(y_p(x)\) is any particular solution to \(\tilde{F}\).

        \end{theorem}

\pagebreak

\section{Lecture 14: February 24, 2023}

    \subsection{Solving Linear Homogeneous Equations With Constant Coefficients: Part I}

        Consider the second order homogeneous linear differential equation with constant coefficients
        \begin{equation*}
            F(x,y,y',y'')=ay''+by'+cy=0,
        \end{equation*}
        for \(a,b,c\in\mathbb{C}\). We may, without loss of generality, rewrite this as
        \begin{equation*}
            \tilde{F}(x,y,y',y'')=y''+Ay'+By=0,
        \end{equation*}
        since \(a\neq0\). Choose some \(\alpha,\beta\in\mathbb{C}\) such that \(-(\alpha+\beta)=A\) and \(\alpha\beta=B\). Now, consider \(z=y'(x)-\alpha y(x)\). Then, note that \(\frac{\dd z}{\dd x}-\beta z=0\). That is,
        \begin{align*}
            0&=y''-\alpha y'(x)-\beta y'(x)+\alpha\beta y(x) \\
            &=y''-(\alpha+\beta)y'+2\beta y \\
            &=y''+Ay'+By.
        \end{align*}
        Now, since \(\frac{\dd z}{\dd x}-\beta z=0\), we have, by separation of variables, \(z=c_1e^{\beta x}\) where \(c_1\in\mathbb{C}\) and \(x\in\mathbb{R}\).  Now, we have
        \begin{equation*}
            z=y'(x)-\alpha y(x)=c_1e^{\beta x}.
        \end{equation*}
        We may use the integrating factor \(\mu(x)=e^{\int -\alpha \dd x}=e^{-\alpha x}\). Consider
        \begin{align*}
            y'(x)e^{-\alpha x}-\alpha y(x)e^{-\alpha x}&=c_1e^{\beta x}e^{-\alpha x} \\
            &=\frac{\dd}{\dd x}(y(x)e^{-\alpha x}).
        \end{align*}
        By integration,
        \begin{align*}
            y(x)&=e^{\alpha x}\int c_1e^{(\beta-\alpha) x} \dd x \\
            &=
            \begin{cases}
                e^{\alpha x}(c_1x+c_2)=c_1xe^{\alpha x}+c_2e^{\alpha x} & \alpha=\beta \\
                e^{\alpha x}\left(\frac{c_1e^{(\beta-\alpha)x}}{\beta-\alpha}+c_2\right)=\frac{c_1}{\beta-\alpha}e^{\beta x}+c_2e^{\alpha x}=\tilde{c}_1e^{\beta x}+c_2e^{\alpha x} & \alpha\neq\beta
            \end{cases}.
        \end{align*}
        In the first case \(\alpha=\beta\), we have that \(\alpha=\beta\) is a double root of the polynomial \(m^2-2\alpha m+\alpha^2=0\). In the second case, \(\alpha\) and \(\beta\) are distinct solutions to the polynomial \(m^2-(\alpha+\beta)m+\alpha\beta=0\). Both solutions are truly general solutions, because there are \(n=2\) linearly independent terms in the linear combination.  
        \pagebreak

\section{Lecture 15: February 28, 2023}

    \subsection{Solving Linear Homogeneous Equations With Constant Coefficients: Part II}

        With the discoveries in the previous section, we provide the following definition and theorem.
        \begin{definition}{\Stop\,\,Characteristic Polynomials}{characteristicpolynomial}

            Let \(F(x,y,\ldots,y^{(n)})_{\text{hom}}=a_ny^{(n)}+\cdots+a_0y=0\) be an ordinary differential equation. The characteristic polynomial of \(F\) is given by
            \begin{equation*}
                p_F(m)=a_nm^n+\cdots+a_0m^0.
            \end{equation*}
            
        \end{definition}
        \begin{theorem}{\Stop\,\,Finding Solutions With Characteristic Polynomials}{usingcharpoly}

            If \(F(x,y,\ldots,y^{(n)})_{\text{hom}}=a_ny^{(n)}+\cdots+a_0y=0\) has a characteristic polynomial with factors into distinct factors,
            \begin{equation*}
                y(x,c_1,\ldots c_n)=c_1e^{m_1x}+\cdots+c_ne^{m_nx},
            \end{equation*}
            If \(F(x,y,\ldots,y^{(n)})_{\text{hom}}=a_ny^{(n)}+\cdots+a_0y=0\) has a characteristic polynomial that factors into \(k\) distinct roots, \(k<n\), that is,
            \begin{equation*}
                p_F(m)=(m-m_1)^{r_1}+\cdots+(m-m_k)^{r_k},
            \end{equation*}
            where \(r_1+\cdots+r_k=n\). We have
            \begin{equation*}
                y(x,c_1,\ldots,c_n)=(c_1x^0+\cdots+c_{r_1}x^{r_1-1})e^{m_1x}+\cdots+(c_{n-r_k}x^0+\cdots+c_nx^{n-r_k})e^{m_nx}.
            \end{equation*}
            
        \end{theorem}
        \vphantom
        \\
        \\
        We will now provide an important remark about Theorem \ref{thm:usingcharpoly} for the ease of understanding. Let \(p_F(m)\) be a characteristic polynomial with root \(m_k\in\mathbb{C}\). We may use Euler's Formula to remove imaginary exponents from the \(n\)th parameter family given by Theorem \ref{thm:usingcharpoly}. For convenience, a statement of Euler's Formula is given in Theorem \ref{thm:eulerformula}.
        \pagebreak
        \begin{theorem}{\Stop\,\,Euler's Formula}{eulerformula}

            For \(\theta\in\mathbb{R}\),
            \begin{equation*}
                e^{i\theta}=\cos\theta+i\sin\theta.
            \end{equation*}
            \begin{proof}
                Recall that \(i^0=i^4=i^8=\cdots=1\), \(i^1=i^5=i^9=\cdots=i\), \(i^2=i^6=i^10=\cdots=-1\), and \(i^3=i^7=i^{11}=\cdots=-i\). If we define, for \(z\in\mathbb{C}\), 
                \begin{equation*}
                    e^z=\sum_{k=0}^\infty \frac{z^k}{k!},
                \end{equation*}
                we have that
                \begin{align*}
                    e^{i\theta}&=1+ix+\frac{(ix)^2}{2!}+\frac{(ix)^3}{3!}+\frac{(ix)^4}{4!}+\frac{(ix)^5}{5!}+\cdots \\
                    &=1+ix-\frac{x^2}{2!}-\frac{ix^3}{3!}+\frac{x^4}{4!}+\frac{ix^5}{5!}+\cdots \\
                    &=\left(1-\frac{x^2}{2!}+\frac{x^4}{4!}+\cdots\right)+i\left(x-\frac{x^3}{3!}+\frac{x^5}{5!}+\cdots\right) \\
                    &=\cos\theta+i\sin\theta,
                \end{align*}
                as desired.
            \end{proof}
            
        \end{theorem}
        \vphantom
        \\
        \\
        To use Euler's Formula, suppose that \(p_F(m)\) is our characteristic polynomial with root \(m_k=a+bi\in\mathbb{C}\). Then, \(\bar{m_\ell}=a-bi\in\mathbb{C}\) is also a root of \(p_F(m)\). Without loss of generality, suppose both \(m_k\) and \(m_\ell\) have multiplicity \(1\). Let our solution components be
        \begin{equation*}
            y_k(x)=e^{(a+bi)x}=e^{ax}(\cos(bx)+i\sin(bx)),\quad y_\ell(x)=e^{(a-bi)x}=e^{ax}(\cos(bx)-i\sin(bx)).
        \end{equation*}
        Since the linear combination of \(y_k\) and \(y_\ell\) is a solution, \(\frac{1}{2}y_k(x)+\frac{1}{2}y_\ell(x)=e^{ax}\cos(bx)\) is a solution, and \(\frac{1}{2i}y_k(x)-\frac{1}{2i}y_\ell(x)=e^{ax}\sin(bx)\) is also a solution. Thus,
        \begin{equation*}
            y_{k,\ell}=c_ke^{ax}\cos(bx)+c_\ell e^{ax}\sin(bx)
        \end{equation*}
        is a solution.
        \pagebreak
        \\
        \\
        Consider the following examples. Note that only Examples \ref{exa:mixedp1}, \ref{exa:mixedp2}, and \ref{exa:mixedp3} were worked in Spring 2023, so these three examples are the best examplars. Nonetheless, the other examples are correct, but may be missing small details, such as omitting the interval of solution. 
        \begin{example}{\Difficulty\,\Difficulty\,\,Real Roots 1}{realroots1}
            
            Find a general solution to
            \begin{equation*}
                F(x,y,y',y'')=2y''+y'-6y=0.
            \end{equation*}
            The characteristic polynomial is
            \begin{equation*}
                2m^2+m-6=0,
            \end{equation*}
            which can be rewritten as
            \begin{equation*}
                (2m-3)(m+2)=0.
            \end{equation*}
            We can see that \(m=-2\) and \(m=\frac{3}{2}\) are solutions to the characteristic polynomial. Therefore,
            \begin{equation*}
                y(x)=c_1e^{-2x}+c_2e^{\frac{3}{2}x}.
            \end{equation*}
        
        \end{example}
        \begin{example}{\Difficulty\,\Difficulty\,\,Real Roots 2}{realroots2}
        
            Find a general solution to
            \begin{equation*}
                F(x,y,\ldots,y''')=y'''-6y''+11y'-6y=0.
            \end{equation*}
            The characteristic polynomial is
            \begin{equation*}
                m^3-6m^2+11m-6=0.
            \end{equation*}
            We may carry out polynomial long division, with the substitution \(x=m\), as follows
            \begin{equation*}
                \polylongdiv{x^3-6x^2+11x-6}{x-1}.
            \end{equation*}
            We can see that \(m=1\), \(m=2\), and \(m=3\) are solutions to the characteristic polynomial. Therefore, 
            \begin{equation*}
                v=c_1e^x+c_2e^{2x}+c_3e^{3x}.
            \end{equation*}
        
        \end{example}
        \pagebreak
        \begin{example}{\Difficulty\,\Difficulty\,\,Real Roots 3}{realroots3}
        
            Find a general solution to
            \begin{equation*}
                F(x,y,\ldots,y''')=2y'''-7y''+4y'+4y=0.
            \end{equation*}
            The characteristic polynomial is
            \begin{equation*}
                2m^3-7m^2+4m+4=0.
            \end{equation*}
            We may carry out polynomial long division, with the substitution \(x=m\), as follows
            \begin{equation*}
                \polylongdiv{2x^3-7x^2+4x+4}{x-2}.
            \end{equation*}
            Therefore, the characteristic polynomial may be rewritten as
            \begin{equation*}
                (m-2)(2m^2-3m-2)=0.
            \end{equation*}
            The above quadratic can be factored as
            \begin{equation*}
                (2m+1)(m-2).
            \end{equation*}
            Therefore, the characteristic polynomial is
            \begin{equation*}
                (m-2)^2(2m+1)=0.
            \end{equation*}
            Therefore,
            \begin{equation*}
                y(x)=c_1e^{2x}+c_2xe^{2x}+c_3e^{-\frac{1}{2}x}.
            \end{equation*}
        
        \end{example}
        \pagebreak
        \begin{example}{\Difficulty\,\Difficulty\,\,Real Roots 4}{realroots4}
        
            Find a general solution to
            \begin{equation*}
                F(x,y,\ldots,y'''')=y''''-y'''-13y''+y'+12y=0.
            \end{equation*}
            The characteristic polynomial is
            \begin{equation*}
                m^4-m^3-13m^2+m+12=0.
            \end{equation*}
            We may carry out polynomial long division, with the substitution \(x=m\), as follows
            \begin{equation*}
                \polylongdiv{x^4-x^3-13x^2+x+12}{x-1}.
            \end{equation*}
            Therefore, the characteristic polynomial may be rewritten as
            \begin{equation*}
                (m-1)(m^3-13m-12)=0.
            \end{equation*}
            We may also perform polynomial long division with the above cubic, with the same substitution as before, producing
            \begin{equation*}
                \polylongdiv{x^3-13x-12}{x+1}.
            \end{equation*}
            Therefore, we have
            \begin{equation*}
                (m-1)(m+1)(m-4)(m+3)=0.
            \end{equation*}
            Therefore,
            \begin{equation*}
                y(x)=c_1e^{-x}+c_2e^{x}+c_3e^{4x}+c_4e^{-3x}.
            \end{equation*}
            
        \end{example}
        \pagebreak
        \begin{example}{\Difficulty\,\Difficulty\,\,Real Roots 5}{realroots5}
        
            Find a general solution to
            \begin{equation*}
                F(x,y,\ldots,y''''')=y'''''-4y''''-8y'''+14y''+7y'-10y=0.
            \end{equation*}
            The characteristic polynomial is
            \begin{equation*}
                m^5-4m^4-8m^3+14m^2+7m-10=0.
            \end{equation*}
            We may carry out polynomial long division, with the substitution \(x=m\), as follows
            \begin{equation*}
                \polylongdiv{x^5-4x^4-8x^3+14x^2+7x-10}{x-1}.
            \end{equation*}
            Therefore, the characteristic polynomial may be rewritten as
            \begin{equation*}
                (m-1)(m^4-3m^3-11m^2+3m+10)=0.
            \end{equation*}
            We may also perform polynomial long division with the above quartic, with the same substitution as before, producing
            \begin{equation*}
                \polylongdiv{x^4-3x^3-11x^2+3x+10}{x+2}.
            \end{equation*}
            We then obtain
            \begin{equation*}
                (m-1)(m+2)(m^3-5m^2-m+5)=0.
            \end{equation*}
            We may proceed similarly to produce
            \begin{equation*}
                (m-1)^2(m+2)(m-5)(m+1)=0.
            \end{equation*}
            Therefore,
            \begin{equation*}
                y(x)=c_1e^x+c_2xe^x+c_3e^{-2x}+c_4e^{-x}+c_5e^{5x}.
            \end{equation*}
            
        \end{example}
        \begin{example}{\Difficulty\,\Difficulty\,\,Complex Roots 1}{complexroots1}
        
            Find a general solution to
            \begin{equation*}
                F(x,y,y',y'')=9y''+6y'+4y=0.
            \end{equation*}
            The characteristic polynomial is
            \begin{equation*}
                9m^2+6m+4.
            \end{equation*}
            By the quadratic formula,
            \begin{equation*}
                m=-\frac{1}{3}\pm\frac{\sqrt{3}}{3}.
            \end{equation*}
            Therefore,
            \begin{equation*}
                y(x)=e^{-\frac{1}{3}}\left(c_1\cos\left(x\frac{\sqrt{3}}{3}\right)+c_2\sin\left(x\frac{\sqrt{3}}{3}\right)\right).
            \end{equation*}
            
        \end{example}
        \begin{example}{\Difficulty\,\Difficulty\,\,Complex Roots 2}{complexroots2}
        
            Find a general solution to
            \begin{equation*}
                F(x,y,\ldots,y''')=y'''+27y'=0.
            \end{equation*}
            The characteristic polynomial is
            \begin{equation*}
                m^3+27m=0,
            \end{equation*}
            or
            \begin{equation*}
                m(m^2+27)=0,
            \end{equation*}
            which can further be factored as
            \begin{equation*}
                m(m+3i\sqrt{3})(m-3i\sqrt{3})=0.
            \end{equation*}
            Solutions are then \(m=0\) and \(m=\pm3i\sqrt{3}\). Therefore,
            \begin{equation*}
                y(x)=c_1+c_2\cos(3x\sqrt{3})+c_3\sin(3x\sqrt{3}).
            \end{equation*}
        
        \end{example}
        \pagebreak
        \vphantom
        \\
        \\
        Consider the following mixed practice exercises.
        \begin{example}{\Difficulty\,\Difficulty\,\,Mixed Practice With Characteristic Polynomials 1}{mixedp1}
            
            Find a general solution to
            \begin{equation*}
                F(x,y,\ldots,y''')=2y'''+3\sqrt{2}y''-4y'=0.
            \end{equation*}
            The characteristic polynomial is
            \begin{align*}
                p_F(m)&=2m^3+3\sqrt{2}m^2-4m \\
                &=m(2m^2-2\sqrt{2}m-4) \\
                &=m(2m-\sqrt{2})(m+\sqrt{8})=0.
            \end{align*}
            Therefore,
            \begin{equation*}
                y(x)=c_1+c_2e^{\frac{x\sqrt{2}}{2}}+c_3e^{-x\sqrt{8}}
            \end{equation*}
            on \(\mathbb{R}\).

        \end{example}
        \begin{example}{\Difficulty\,\Difficulty\,\,Mixed Practice With Characteristic Polynomials 2}{mixedp2}
            Find a general solution to
            \begin{equation*}
                F(x,y,\ldots,y''''')=y'''''-2y''''+y'''=0.
            \end{equation*}
            The characteristic polynomial is
            \begin{align*}
                p_F(m)&=m^5-2m^4+m^3 \\
                &=m^3(m^2-2m+1) \\
                &=m^3(m-1)^2=0.
            \end{align*}
            Therefore,
            \begin{equation*}
                y(x)=c_1+c_2x+c_3x^2+c_4e^{x}+c_5xe^x
            \end{equation*}
            on \(\mathbb{R}\).
        \end{example}
        \pagebreak
        \begin{example}{\Difficulty\,\Difficulty\,\,Mixed Practice With Characteristic Polynomials 3}{mixedp3}
            Find a general solution to
            \begin{equation*}
                F(x,y,\ldots,y''')=y'''+3y''+3y'+2y=0.
            \end{equation*}
            The characteristic polynomial is
            \begin{align*}
                p_F(m)&=m^3+3m^2+3m+2 \\
                &=(m+2)(m^2+m+1) \\
                &=(m+2)\left(m-\left(\frac{-1+i\sqrt{3}}{2}\right)\right)\left(m-\left(\frac{-1-i\sqrt{3}}{2}\right)\right)
            \end{align*}
            Therefore,
            \begin{equation*}
                y(x)=c_1e^{-2x}+e^{-\frac{x}{2}}\left(c_2\cos\left(\frac{x\sqrt{3}}{2}\right)+c_2\sin\left(\frac{x\sqrt{3}}{2}\right)\right)
            \end{equation*}
            on \(\mathbb{R}\).
        \end{example}

\pagebreak

\section{Lectures 16, February 27, 2023 \& Lecture 17, March 1, 2023}

    \subsection{The Method of Undetermined Coefficients: Part I}

        Consider \(F(x,y\ldots,y^{(n)})=a_ny^{(n)}+\cdots+a_1y'+a_0y-Q(x)=0\). Recall that the general solution 
        \begin{equation*}
            y(x)=y_c(x)+y_p(x)
        \end{equation*}
        where \(y_c(x)\) is the general solution to \(F(x,y,\ldots,y^{(n)})_{\text{hom}}=0\). We wish to find \(y_p(x)\). If \(Q(x)\) is an elementary function that can be expressed as a sum of terms, each of which has finitely many linearly independent derivatives, we may use the Method of Undetermined Coefficients.
        \\
        \\
        Note that the Method of Undetermined Coefficients will only work if \(Q(x)\) only contains terms of the form \(a,x^k,e^{ax},\sin ax,\cos ax\), and combinations of the previous terms with \(a\in\mathbb{R}\) and \(k\in\mathbb{Z}^+\).
        \\
        \\
        Consider the following cases that often arise with the Method of Undetermined Coefficients. In the Method of Undetermined Coefficients, we compare the terms of \(y_c(x)\) to those of \(Q(x)\) to find \(y_p(x)\). Consider the following theorem.
        \begin{theorem}{\Stop\,\,Cases of the Method of Undetermined Coefficients}{casesofmethundetcoeff}
            
            The following cases describe the Method of Undetermined Coefficients.
            \begin{enumerate}
                \item If no term of the forcing term \(Q(x)\) is the same as a term of \(y_c(x)\), \(y_p(x)\) will be a linear combination of the terms of \(Q(x)\) and all its linearly independent derivatives.
                \item If the forcing term \(Q(x)\) contains a term which, ignoring constant coefficients, is \(x^ku(x)\), where \(u(x)\) is a term of \(y_c(x)\) and \(k\in\mathbb{N}\), \(y_p(x)\) will be a linear combination of \(x^{k+1}u(x)\) and all of its linearly independent derivatives, again ignoring constant coefficients. Note that if \(Q(x)\) contains terms which belong to Case \(1\), include them appropriately in \(y_p(x)\).
            \end{enumerate}
            
        \end{theorem}
        \pagebreak
        \vphantom
        \\
        \\
        For the Method of Undetermined Coefficients, to form the necessary linear combinations, consider the following table.
        \begin{table}[h!]
            \centering
            \begin{tabular}{|c|c|}
                \hline
                \hspace{5em} & \hspace{15em} \\
                \textbf{Forcing Term \(Q(x)\)} & \textbf{General Form of \(y_p(x)\)} \\
                & \\ \hline \hline
                & \\
                \(ae^{\beta x}\) & \(Ae^{\beta x}\) \\
                & \\ \hline
                & \\
                \(a\cos(\beta x)+b\sin(\beta x)\) & \(A\cos(\beta x)+B\sin(\beta x)\) \\
                & \\ \hline
                & \\
                \(a_nx^n+a_{n-1}x^{n-1}+\cdots+a_1x+a_0\) & \(A_nx^n+A_{n-1}x^{n-1}+\cdots+A_1+A_0\) \\
                & \\ \hline
                & \\
                \(e^{\beta x}(a_nx^n+a_{n-1}x^{n-1}+\cdots+a_1x+a_0)\) & \(e^{\beta x}(A_nx^n+A_{n-1}x^{n-1}+\cdots+A_1+A_0)\) \\
                & \\ \hline
                & \\
                \(\cos(\beta x)(a_nx^n+a_{n-1}x^{n-1}+\cdots+a_1x+a_0)\) & \(\cos(\beta x)(A_nx^n+A_{n-1}x^{n-1}+\cdots+A_1+A_0)\) \\
                & \\ \hline
                & \\
                \(\sin(\beta x)(a_nx^n+a_{n-1}x^{n-1}+\cdots+a_1x+a_0)\) & \(\sin(\beta x)(A_nx^n+A_{n-1}x^{n-1}+\cdots+A_1+A_0)\) \\
                & \\ \hline
            \end{tabular}
            \caption{Table of the General Forms of \(y_p(x)\) for the Method of Undetermined Coefficients}
            \label{table:muc}
        \end{table}
\pagebreak

\section{Lecture 18, March 3, 2023}

    \subsection{The Method of Undetermined Coefficients: Part II}

    Theorem \ref{thm:casesofmethundetcoeff} is quite technical, so we provide the following examples.
    \begin{example}{\Difficulty\,\Difficulty\,\,Case 1 of the Method of Undetermined Coefficients 1}{case1methundetcoeff1}
        
        FInd a general solution to \(F(x,y,y',y'')=y''+4y'+4y=4x^2+6e^x\) given that 
        \begin{equation*}
            y_c(x)=c_1e^{-2x}+c_2xe^{-2x}.
        \end{equation*}
        By Theorem \ref{thm:casesofmethundetcoeff}, \(y_p(x)\) is a linear combination of the terms of \(Q(x)=4x^2+6e^x\) and its derivatives. Suppose, then, that \(y_p(x)=Ax^2+Bx+Ce^x+D\). Then, \(y_p'(x)=2Ax+Ce^x+B\) and \(y_p''(x)=Ce^x+2A\). Then,
        \begin{align*}
            F(x,y_p,y_p',y_p'')&=Ce^x+2A+4(2Ax+Ce^x+B)+4(Ax^2+Bx+Ce^x+D)&=4x^2+6e^x \\
            &=Ce^x+2A+8Ax+4Ce^x+4B+4Ax^2+4Bx+4Ce^x+4D&=4x^2+6e^x \\
            &=4Ax^2+(8A+4B)x+9Ce^x+2A+4B+4D&=4x^2+6e^x.
        \end{align*}
        By matching coefficients, \(4Ax^2=4x^2\), so \(A=1\). Then, \(8A+4B=0\), so \(B=-2\). Then, \(9Ce^x=6\), so \(C=\frac{2}{3}\). Finally, since \(2A+4B+4D=0\), so \(D=\frac{3}{2}\). Therefore, our general solution is
        \begin{equation*}
            y(x)=c_1e^{-2x}+c_2xe^{-2x}+x^2-2x+\frac{2}{3}e^x+\frac{3}{2}
        \end{equation*}
        on \(\mathbb{R}\).

    \end{example}
    \begin{example}{\Difficulty\,\Difficulty\,\,Case 1 of the Method of Undetermined Coefficients 2}{case1methundetcoeff2}
        
        Find a general solution to \(F(x,y,y',y'')=y''-3y'+2y=2xe^{3x}+3\sin x\) given that
        \begin{equation*}
            y_c(x)=c_1e^{3x}+c_2e^{2x}
        \end{equation*}
        By Theorem \ref{thm:casesofmethundetcoeff}, \(y_p(x)\) is a linear combination of the terms of \(Q(x)=2xe^{3x}+3\sin x\) and its derivatives. Suppose, then, that \(y_p(x)=A\sin x+B\cos x+Ce^{3x}+Dxe^{3x}\). Then, \(y_p'(x)=A\cos x-B\sin x+3Ce^{3x}+3Dxe^{3x}+De^{3x}\) and \(y_p''(x)=-A\sin x-B\cos x+9Ce^{3x}+9Dxe^{3x}+6De^{3x}\). Then,
        \begin{align*}
            F(x,y_p,y_p',y_p'')&=2Dxe^{3x}+(2C+3D)e^{3x}+(A+3B)\sin x+(B-3A)\cos x &= 2xe^{3x}+3\sin x.
        \end{align*}
        By matching coefficients, \(2Dx=2x\), so \(D=1\). Then, \(2C+3D=0\), so \(C=-\frac{3}{2}\). We also have \(A+3B=3\) and \(B-3A=0\), so \(A=\frac{3}{10}\) and \(B=\frac{9}{10}\). Therefore, our general solution is
        \begin{equation*}
            y(x)=c_1e^{3x}+c_2e^{2x}+\frac{3}{10}\sin x+\frac{9}{10}\cos x-\frac{3}{2}e^{3x}+xe^{3x}
        \end{equation*}
        on \(\mathbb{R}\).

    \end{example}
    \begin{example}{\Difficulty\,\Difficulty\,\,Case 1 of the Method of Undetermined Coefficients 3}{case1methundetcoeff3}
        
        Find a general solution to \(F(x,y,y',y'')=y''-y=5x\).
        \\
        \\
        We see that \(p_F(m)=m^2-1\), so \(y_c(x)=c_1e^x+c_2e^{-x}\). Then, by Theorem \ref{thm:casesofmethundetcoeff}, \(y_p(x)\) is a linear combination of \(Q(x)=5x\) and its derivatives. Suppose, then, that \(y_p(x)=Ax+B\). Then, \(y_p'(x)=A\) and \(y_p''(x)=0\). Then,
        \begin{align*}
            F(x,y_p,y_p',y_p'')&=0-Ax-B&=5x.
        \end{align*}
        Thus, \(A=-5\) and \(B=0\). Therefore, our general solution is
        \begin{equation*}
            y(x)=c_1e^x+c_2e^{-x}-5x
        \end{equation*}
        on \(\mathbb{R}\).

    \end{example}
    \begin{example}{\Difficulty\,\Difficulty\,\,Case 2 of the Method of Undetermined Coefficients 1}{case1methundetcoeff}
        
        Find a general solution to \(F(x,y,y',y'')=y''+y=\sin x\) given that
        \begin{equation*}
            y_c(x)=c_1\cos x+c_2\sin x.
        \end{equation*}
        By Theorem \ref{thm:casesofmethundetcoeff}, since \(\sin x=x^0\sin x\), \(y_p(x)\) will be a linear combination of \(x\sin x\) and its linearly independent derivatives. We have that \(y_p(x)=Ax\sin x+Bx\cos x\). Then, \(y_p'(x)=A\sin x+Ax\cos x-Bx\sin x+B\cos x\) and \(y_p''(x)=A\cos x+A\cos x-Ax\sin x-Bx\cos x-B\sin x-B\sin x\). Then,
        \begin{align*}
            F(x,y_p,y_p',y_p'')&=2A\cos x-2B\sin x&=\sin x.
        \end{align*}
        Thus, \(A=0\) and \(B=-\frac{1}{2}\). Therefore, our general solution is
        \begin{equation*}
            y(x)=c_1\cos x+c_2\sin x-\frac{1}{2}x\cos x
        \end{equation*}
        on \(\mathbb{R}\).

    \end{example}
    \pagebreak
    \begin{example}{\Difficulty\,\Difficulty\,\,Case 2 of the Method of Undetermined Coefficients 2}{case1methundetcoeff}

        Find a general solution to \(y'''-4y''+4y'=3x^3\) given that
        \begin{equation*}
            y_c(x)=c_1+c_2e^{2x}+c_3xe^{2x}.
        \end{equation*}
        By Theorem \ref{thm:casesofmethundetcoeff}, since \(3x^3=c_1x^3(1)\) ignoring constant coefficients, \(y_p(x)\) wil be a linear combination of \(x^4\) and all its linearly independent derivatives. We have that \(y_p(x)=Ax^4+Bx^3+Cx^2+Dx+E\). Then \(y_p'(x)=4Ax^3+3Bx^2+2Cx+D\), \(y_p''(x)=12Ax^2+6Bx+2C\), and \(y_p'''(x)=24Ax+6B\). Then,
        \begin{align*}
            F(x,y_p,y_p',y_p'',y_p''')&=24Ax+6B-48Ax^2-24Bx-8C+16Ax^3+12Bx^2+8Cx+4D&=3x^3 \\
            &=(6B-8C+4D)+(24A-24B+8C)x+(-48A+12B)x^2+(16A)x^3&=3x^3.
        \end{align*}
        Thus, \(A=\frac{3}{16}\), \(B=\frac{3}{4}\), \(C=\frac{27}{16}\), and \(D=\frac{9}{4}\). Therefore, our general solution is
        \begin{equation*}
            y(x)=c_1+c_2e^{2x}+c_3xe^{2x}+\frac{3}{16}x^4+\frac{3}{4}x^3+\frac{27}{16}x^2+\frac{9}{4}x
        \end{equation*}
        on \(\mathbb{R}\).
        
    \end{example}

\pagebreak

\section{Lecture 19, March 6, 2023}

    \subsection{Variation of Parameters: Part I}

        Recall that the Method of Undetermined Coefficients only works if the forcing term \(Q(x)\) in
        \begin{equation*}
            F(x,y,\ldots,y^{(n)})=f_n(x)y^{(n)}+\cdots+f_1(x)y'+f_0(x)y-Q(x)=0
        \end{equation*}
        can be expressed as a sum of terms, each of which has finitely many linearly independent derivatives. This severely restricts the solution technique, as \(x^{-n},\log(ax),\tan(x),\ldots\) have infinitely many derivatives. Consider the following outline.
        \begin{enumerate}
            \item Suppose \(F(x,y,\ldots,y^{(n)})=0\) has associated \(F(x,y,\ldots,y^{(n)})_{\text{hom}}=0\) with general solution \(y_c(x)=c_1y_1(x)+\cdots+c_ny_n(x)\).
            \item Suppose that \(y_p(x)\) has the form \(y_p(x)=u_1(x)y_1(x)+\cdots+u_n(x)y_n(x)\) where \(u_1(x),\ldots,u_n(x)\) are continuously differentiable. If we take \(n\) derivatives and substitute into our original differential equation, we can form the system of \(n\) equations
            \begin{align*}
                \begin{cases}
                    u_1'(x)y_1(x)+\cdots+u_n'(x)y_n(x)&=0 \\
                    u_1'(x)y_1'(x)+\cdots+u_n'(x)y_n'(x)&=0 \\
                    &\vdots \\
                    u_1'(x)y_1^{(n-2)}(x)+\cdots+u_n'(x)y_n^{(n-2)}(x)&=0 \\
                    u_1'(x)y_1^{(n-1)}(x)+\cdots+u_n'(x)y_n^{(n-1)}(x)&=\frac{Q(x)}{f_n(x)}
                \end{cases}.
            \end{align*}
        \end{enumerate}
        \vphantom
        \\
        \\
        We wish to find \(u_i'(x)\) such that it is possible to solve for \(u_i(x)=\int u_i'(x)\dd x\). Consider
        \begin{equation*}
            \begin{bmatrix}
                    y_1(x) & \cdots & y_n(x) \\
                    y_1'(x) & \cdots & y_n'(x) \\
                    \vdots & \ddots & \vdots \\
                    y_1^{(n-2)}(x) & \cdots & y_n^{(n-2)}(x) \\
                    y_1^{(n-1)}(x) & \cdots & y_n^{(n-1)}(x)
            \end{bmatrix}
            \begin{bmatrix}
                u_1'(x) \\ u_2'(x) \\ \vdots \\ u_{n-1}'(x) \\ u_n'(x) 
            \end{bmatrix}
            =
            \begin{bmatrix}
                0 \\ 0 \\ \vdots \\ 0 \\ \frac{Q(x)}{f_n(x)}
            \end{bmatrix}
        \end{equation*}
        We can solve this system using Cramer's Rule or matrix inverses. This is possible since \(S=\{y_1\ldots,y_n\}\) is linearly independent, so \(W(S)(x)\neq0\) on \(I\). Thus, our coefficient matrix is invertible and \(\det A\neq0\). For convenience, Cramer's Rule is provided below.
        \begin{theorem}{\Stop\,\,Cramer's Rule}{cramersrule}

            Suppose we have a system \(A\vec{X}=\vec{B}\), and \(\det A\neq0\). Let \(\vec{X}=[x_1,\ldots,x_n]^T\) be the solution vector. Then,
            \begin{equation*}
                x_i=\frac{\det A_i}{\det A}
            \end{equation*}
            where \(A_i\) is obtained by replacing the \(i\)th column of \(A\) with \(\vec{b}\).
            
        \end{theorem}
        \pagebreak
        \vphantom
        \\
        \\
        Consider the following example.
        \begin{example}{\Difficulty\,\Difficulty\,\,Variation of Parameters 1}{varparam1}

            Find a solution component of \(y^{(5)}-2y^{(4)}+y'''+x-e^{-x}=0\).
            \\
            \\
            Note that here, \(Q(x)=e^{-x}-x\) and \(f_5(x)=1\). Note
            \begin{equation*}
                y_c(x)=c_1y_1(x)+c_2y_2(x)+c_3y_3(x)+c_4y_4(x)+c_5y_5(x)=c_1+c_2x+c_3x^2+c_4e^x+c_5xe^x.
            \end{equation*}
            Thus,
            \begin{equation*}
                A=\begin{bmatrix}
                    1 & x & x^2 & e^x & xe^x \\
                    0 & 1 & 2x & e^x & e^x+xe^x \\
                    0 & 0 & 2 & e^x & 2e^x+xe^x \\
                    0 & 0 & 0 & e^x & 3e^x+xe^x \\
                    0 & 0 & 0 &  e^x & 4e^x+xe^x
                \end{bmatrix}\underbrace{\to}_{\text{REF}}\begin{bmatrix}
                    1 & x & x^2 & e^x & xe^x \\
                    0 & 1 & 2x & e^x & e^x+xe^x \\
                    0 & 0 & 2 & e^x & 2e^x+xe^x \\
                    0 & 0 & 0 & e^x & 3e^x+xe^x \\
                    0 & 0 & 0 & 0 & -3e^x-xe^x+4e^x+xe^x
                \end{bmatrix}
            \end{equation*}
            Note that
            \begin{equation*}
                \vec{b}=\begin{bmatrix}
                    0 \\ 0 \\ 0 \\ 0 \\ e^{-x}-x
                \end{bmatrix}.
            \end{equation*}
            Since \(\ef A \in\mathcal{U}\), and our row operation didn't change the value of the determinant of \(A\), we have that \(\det A=2e^{2x}\). By Cramer's Rule,
            \begin{align*}
                u_5'(x)=\frac{\det A_5}{\det A}&=\frac{1}{2e^{2x}}\det \begin{bmatrix}
                1 & x & x^2 & e^x & 0 \\
                0 & 1 & 2x & e^x & 0 \\
                0 & 0 & 2 & e^x & 0 \\
                0 & 0 & 0 & e^x & 0 \\
                0 & 0 & 0 &  e^x & e^{-x}-x
            \end{bmatrix} \\
            &=\frac{1}{2e^{2x}}\det \begin{bmatrix}
                1 & x & x^2 & e^x & 0 \\
                0 & 1 & 2x & e^x & 0 \\
                0 & 0 & 2 & e^x & 0 \\
                0 & 0 & 0 & e^x & 0 \\
                0 & 0 & 0 & 0 & e^{-x}-x
            \end{bmatrix} \\
            &=\frac{2e^x(e^{-x}-x)}{2e^{2x}}=\frac{1-xe^{x}}{e^{2x}}=e^{-2x}-xe^{-x}.
            \end{align*}
            Then, we have that
            \begin{equation*}
                u_5(x)=\int (e^{-2x}-xe^{-x})\dd x.
            \end{equation*}
        \end{example}

\section{Lecture 20, March 8, 2023}

    \subsection{Variation of Parameters: Part II}

        Consider the following examples.
        \begin{example}{\Difficulty\,\Difficulty\,\,Variation of Parameters 2}{varparam2}
            
            Find a general solution to \(F(x,y,y',y'')=y''-2y'-\log x=0\).
            \\
            \\
            Here \(Q(x)=\log x\). Note that we cannot use the method of Undetermined Coefficients, as \(\log x\) has infinitely many derivatives. Since \(F(x,y,y',y'')_\text{hom}=y''-2y'=0\), \(p_F(m)=m^2-2m=0\), so
            \begin{equation*}
                y_c(x)=c_1+c_2e^{2x}.
            \end{equation*}
            Thus,
            \begin{equation*}
                A=\begin{bmatrix}
                    1 & e^{2x} \\
                    0 & 2e^{2x}
                \end{bmatrix}.
            \end{equation*}
            Note that
            \begin{equation*}
                \vec{b}=\begin{bmatrix}
                    0 \\
                    \log x
                \end{bmatrix}.
            \end{equation*}
            We have that \(\det A=2e^{2x}\), so by Cramer's Rule,
            \begin{align*}
                u_1'(x)=\frac{\det A_1}{\det A}&=\frac{1}{2e^{2x}}\det \begin{bmatrix}
                    0 & e^{2x} \\
                    \log x & 2e^{2x}
                \end{bmatrix} \\
                &=\frac{-e^{2x}\log x}{2e^{2x}}=-\frac{\log x}{2}
            \end{align*}
            and
            \begin{align*}
                u_2'(x)=\frac{\det A_2}{\det A}&=\frac{1}{2e^{2x}}\det \begin{bmatrix}
                    1 & 0 \\
                    0 & \log x
                \end{bmatrix} \\
                &=\frac{\log x}{2e^{2x}}.
            \end{align*}
            Then,
            \begin{align*}
                y_p(x)&=-\frac{1}{2}\int \log x \dd x+e^{2x}\int \frac{\log x}{e^{2x}}\dd x \\
                &=-\frac{1}{2}\left(x\log x-x\right)+\frac{e^{2x}}{2}\int \frac{\log x}{e^{2x}}\dd x.
            \end{align*}
            Therefore,
            \begin{equation*}
                y(x)=c_1+c_2e^{2x}=-\frac{1}{2}\left(x\log x-x\right)+\frac{e^{2x}}{2}\int \frac{\log x}{e^{2x}}\dd x,
            \end{equation*}
            on \(\{x\in\mathbb{R}:x>0\}\).

        \end{example}
        \pagebreak
        \begin{example}{\Difficulty\,\Difficulty\,\,Variation of Parameters 3}{varparam3}

            Find a general solution to \(y''+y=\log(5x+2)\). 
            \\
            \\
            Here \(Q(x)=\log(5x+2)\). Note that we cannot use the Method of Undetermined Coefficients, for a similar reason as Example \ref{exa:varparam2}. We note that \(y_c(x)=c_1\cos x+c_2\sin x\). Then,
            \begin{equation*}
                A=\begin{bmatrix}
                    \cos(x) & \sin(x) \\
                    -\sin(x) & \cos(x)
                \end{bmatrix}.
            \end{equation*}
            Note that
            \begin{equation*}
                \vec{b}=\begin{bmatrix} 0 \\ \log(5x+2) \end{bmatrix}.
            \end{equation*}
            We have that \(\det A=2e^{2x}\), so by Cramer's Rule,
            \begin{equation*}
                u_1'(x)=\frac{1}{\det A}\det\begin{bmatrix}
                    0 & \sin(x) \\
                    \log(5x+2) & \cos(x)
                \end{bmatrix}=-\sin(x)\log(5x+2)
            \end{equation*}
            and
            \begin{equation*}
                u_2'(x)=\frac{1}{\det A}\begin{bmatrix}
                    \cos(x) & 0 \\
                    -\sin(x) & \log(5x+2)
                \end{bmatrix}=\cos(x)\log(5x+2).
            \end{equation*}
            Then,
            \begin{equation*}
                y_p(x)=-\cos(x)\int \sin(x)\log(5x+2)\dd x+\sin(x)\int \cos(x)\log(5x+2) \dd x.
            \end{equation*}
            Therefore,
            \begin{align*}
                y(x)=c_1\cos(x)+c_2\sin(x)-\cos(x)\int \sin(x)\log(5x+2)\dd x+\sin(x)\int \cos(x)\log(5x+2) \dd x
            \end{align*}
            valid on \(\left\{x\in\mathbb{R}:x>-\frac{2}{5}\right\}\).
            
        \end{example}

        \pagebreak

    \subsection{Reduction of Order}

        Consider the following theorems.
        \begin{theorem}{\Stop\,\,A Useful Lemma for Reduction of Order}{usefullemmareduction}

            Given \(n-1\) linearly independent solutions of an \(n\)th order linear differential equation, the \(n\)th \(1\)-parameter family of solutions can be obtained.
            
        \end{theorem}
        \begin{theorem}{\Stop\,\,Reduction of Order, in the \(n=2\) Case}{redord2order}

            Let \(F(x,y,y,',y'')=f_2(x)y''+f_1(x)y'+f_0(x)y-Q(x)=0\) has the known \(1\)-parameter family \(c_1y_1(x)\), a linearly independent second solution is given by
            \begin{equation*}
                y_2(x)=y_1(x)\int \frac{e^{-\int \frac{f_1(x)}{f_2(x)}\dd x}}{(y_1(x))^2} \dd x.
            \end{equation*}
            
        \end{theorem}
        \vphantom
        \\
        \\
        \begin{example}{\Difficulty\,\Difficulty\,\,Reduction of Order 1}{redord1}

            Use Reduction of Order to find the general solution to \(x^2y''-2xy'+2y=0\) given that \(y_1(x)=x\).
            \\
            \\
            By Theorem \ref{thm:redord2order},
            \begin{align*}
                y_2(x)&=x\int \frac{e^{-\frac{-2x}{x^2}}}{x^2}\dd x \\
                &=x\int \frac{e^{\frac{2}{x}}}{x^2} \dd x \\
                &=x\int \dd x \\
                &=x^2,
            \end{align*}
            so \(y_c(x)=c_1x+c_2x^2\) on \(\{x\in\mathbb{R}:x\neq0\}\).
            
        \end{example}
        \pagebreak
        \vphantom
        \\
        \\
        We may now use Variation of Parameters on Example \ref{exa:redord1} with any forcing term \(Q(x)\). Consider the following example.
        \begin{example}{\Difficulty\,\Difficulty\,\,Variation of Parameters With Reduction of Order}{varparamredord}

            Find a general solution to \(x^2y''-2xy'+2y=x\log x\). Here, \(Q(x)=x\log x\), and recall \(y_c(x)=c_1x+c_2x^2\). Thus,
            \begin{equation*}
                A=\begin{bmatrix}
                    x & x^2 \\
                    1 & 2x
                \end{bmatrix}.
            \end{equation*}
            Note that
            \begin{equation*}
                \vec{b}=\begin{bmatrix} 0 \\ \frac{x\log x}{x^2} \end{bmatrix}.
            \end{equation*}
            We have that \(\det A=x^2\), so by Cramer's Rule,
            \begin{align*}
                u_1'(x)&=\frac{1}{x^2}\det\begin{bmatrix} 0 & x^2 \\ \frac{x\log x}{x^2} & 2x \end{bmatrix} \\
                &=-\frac{1}{x}\log x
            \end{align*}
            and
            \begin{align*}
                u_2'(x)&=\frac{1}{x^2}\det\begin{bmatrix} x & 0 \\ 1 & \frac{x\log x}{x^2} \end{bmatrix} \\
                &=\frac{1}{x^2}\log x.
            \end{align*}
            Then,
            \begin{align*}
                y_p(x)&=x\int -\frac{1}{x}\log x \dd x+x^2\int \frac{1}{x^2}\log x\dd x \\
                &=-\frac{x(\log x)^2}{2}+x(-\log x+1)
            \end{align*}
            Thus, the general solution is
            \begin{equation*}
                y(x)=c_1+c_2x-\frac{x(\log x)^2}{2}+x(-\log x+1).
            \end{equation*}
            
        \end{example}

\pagebreak

\section{Lecture 21: March 10, 2023}

    \subsection{The Laplace Transform}

        Recall the definition of an improper integral.
        \begin{definition}{\Stop\,\,Improper Integrals}{impropint}

            Let \(f\in C(\{x\in\mathbb{R}:x\geq \alpha, \alpha\in\mathbb{R}\})\). Then,
            \begin{equation*}
                \int_\alpha^\infty f(x)\dd x=\lim_{\beta\to \infty}\int_\alpha^\beta f(x)\dd x
            \end{equation*}
            exists if and only if the limit exists, and is finite.
            
        \end{definition}
        \vphantom
        \\
        \\
        Consider the following definition of the Laplace Transform and its related theorems.
        \begin{definition}{\Stop\,\,The Laplace Transform}{laplace}
        
            Let \(s_0>0\in\mathbb{R}\) and \(f\in C(\{t\in\mathbb{R}:x\geq s_0\})\). Then, the Laplace Transform of \(f\) is defined as
            \begin{equation*}
                F(s)=\laplace{f(t)}=\int_{t=0}^{t=\infty} e^{-st}f(t)\dd t.
            \end{equation*}
            The improper integral may diverge. To alleviate this, we restrict the Laplace Transform to functions \(f(t)\) of exponential growth order or less.
        
        \end{definition}
        \begin{theorem}{\Stop\,\,The Linearity of the Laplace Transform}{laplaceprop2}
    
            Let \(f(t)\) and \(g(t)\) be functions such that \(\laplace{f(t)}=F(s)\) exists for \(s>s_1\) and \(\laplace{g(t)}=G(s)\) exists for \(s>s_2\) and \(\alpha,\beta\in\mathbb{C}\). Then, choosing \(s>\max(\{s_1,s_2\})\),
            \begin{equation*}
                \laplace{\alpha f(t)+\beta g(t)}=\alpha\laplace{f(t)}+\beta\laplace{g(t)}.
            \end{equation*}
            \begin{proof}
                The result follows from the linearity of the Riemann integral.
            \end{proof}
        
        \end{theorem}
        \pagebreak
        \vphantom
        \\
        \\
        To compute Laplace Transforms, we can use Definition \ref{def:laplace}, but in practice, it is much easier to use a table. A bare-bones table is provided in Table \ref{table:laplace}. Consider the following examples.
        \begin{example}{\Difficulty\,\Difficulty\,\,Computing a Laplace Transform 1}{computinglaplace1}
            
            Compute \(\laplace{6e^{-5t}+e^{3t}+5t^3-9}\).
            \\
            \\
            We have that
            \begin{align*}
                \laplace{6e^{-5t}+e^{3t}+5t^3-9}&=6\laplace{e^{-5t}}+\laplace{e^{3t}}+5\laplace{t^3}-\laplace{9} \\
                &=\frac{6}{s+5}+\frac{1}{s-3}+\frac{30}{s^4}-\frac{9}{s}.
            \end{align*}

        \end{example}
        \begin{example}{\Difficulty\,\Difficulty\,\,Computing a Laplace Transform 2}{computinglaplace2}
            
            Compute \(\laplace{t^2\sin 2t}\).
            \\
            \\
            We have that
            \begin{align*}
                \laplace{t^2\sin 2t}&=\frac{\dd^2}{\dd s^2}\frac{2}{s^2+4} \\
                &=-\frac{\dd}{\dd s}\frac{4s}{(s^2+4)^2} \\
                &=-\left(\frac{4}{(s^2+4)^2}-\frac{16s^2}{(s^2+4)^3}\right).
            \end{align*}

        \end{example}
        \begin{example}{\Difficulty\,\Difficulty\,\,Computing a Laplace Transform 3}{computinglaplace3}
            
            Compute \(\laplace{4\cos 4t-9\sin 4t+2\cos 10t}\).
            \\
            \\
            We have that
            \begin{align*}
                \laplace{4\cos 4t-9\sin 4t+2\cos 10t}&=4\laplace{\cos 4t}-9\laplace{\sin 4t}+2\laplace{\cos 10 t} \\
                &=\frac{4s}{s^2+16}-\frac{36}{s^2+16}+\frac{2s}{s^2+100}.
            \end{align*}

        \end{example}
        \begin{example}{\Difficulty\,\Difficulty\,\,Computing a Laplace Transform 4}{computinglaplace4}
            
            Compute \(\laplace{e^{3t}+\cos 6t-e^{3t}\cos 6t}\).
            \\
            \\
            We have that
            \begin{align*}
                \laplace{e^{3t}+\cos 6t-e^{3t}\cos 6t}&=\laplace{e^{3t}}+\laplace{\cos 6t}-\laplace{e^{3t}\cos 6t} \\
                &=\frac{1}{s-3}+\frac{s}{s^2+36}-\frac{s-3}{(s-3)^2+36}
            \end{align*}

        \end{example}

\pagebreak
    
\section{Lecture 22: March 13, 2023}

    \subsection{The Inverse Laplace Transform}

        THe following theorems are both essential in using the Laplace Transform in order to solve differential equations.
        \begin{theorem}{\Stop\,\,The Laplace of a Derivative}{laplaceprop1}
    
            The Laplace Transform turns differentiation into multiplication. That is, 
            \begin{align*}
                \laplace{f'(t)}&=\int_{t=0}^{t=\infty}e^{-st}f'(t)\dd t \\
                &=\left(\frac{f(t)}{e^{st}}\right)_{t=0}^{t=\infty}+s\int_{t=0}^{t=\infty} e^{-st}f(t) \dd t  \\
                &=\lim_{b\to\infty}\left(\frac{f(t)}{e^{sb}}\right)-f(0)+s\int_{t=0}^{t=\infty} e^{-st}f(t) \dd t \\
                &=s\mathcal{L}(f(t))-f(0).
            \end{align*}
            We must note that this is only possible for functions \(f(t)\) of exponential growth order or less.
        
        \end{theorem}
        \begin{theorem}{\Stop\,\,The Injectivity of the Laplace Transform (Lerch's Theorem)}{injectivelaplace}

            Let \(f(t),g(t)\in C(\{x\in\mathbb{R}:x\geq0\})\) such that \(\laplace{f(t)}=\laplace{g(t)}\). Then, \(f(t)=g(t)\) almost everywhere\footnote{\href{https://en.wikipedia.org/wiki/Almost_everywhere}{Measure Theory.}} on \(\{x\in\mathbb{R}:t\geq0\}\).
            
        \end{theorem}
        \vphantom
        \\
        \\
        Theorem \ref{thm:injectivelaplace} tells us that if we have \(F(s)=G(s)\), where \(F\) and \(G\) are Laplace Transforms, the functions that were transformed to get \(F\) and \(G\) are indeed equal. This means we can construct the following definition.
        \begin{definition}{\Stop\,\,The Inverse Laplace Transform}{inverselaplace}

            If \(\laplace{f(t)}=F(s)\), we define
            \begin{equation*}
                \laplace[-1]{F(s)}=f(t).
            \end{equation*}
            
        \end{definition}
        \pagebreak
        \vphantom
        \\
        \\
        When computing the Inverse Laplace Transform, it is again useful to use a table, such as one provided in Table \ref{table:laplace}. It is also critical to use partial fraction decomposition techniques commonly taught in either a precalculus or second-semester calculus course. For convenience, we provide the below table.
        \begin{table}[h!]
            \centering
            \begin{tabular}{||c|c||}
                \hline
                \hspace{5em} & \hspace{15em} \\
                \textbf{Factor} & \textbf{Term} \\
                & \\ \hline \hline
                & \\
                \((ax+b)^n\) & \(\dfrac{A_1}{(ax+b)}+\cdots+\dfrac{A_n}{(ax+b)^n}\) \\
                & \\ \hline
                & \\
                \((ax^2+bx+c)^n\) & \(\dfrac{A_1x+B_1}{ax^2+bx+c}+\cdots+\dfrac{A_nx+B_n}{(ax^2+bx+c)^n}\) \\
                & \\ \hline
            \end{tabular}
            \caption{Table of Partial Fraction Decompositions}
            \label{table:pfds}
        \end{table}
        \vphantom
        \\
        \\
        Consider the following examples.
        \begin{example}{\Difficulty\,\Difficulty\,\,Computing an Inverse Laplace Transform 1}{computinginvlaplace1}

            Compute \(\laplace[-1]{\frac{6}{s}-\frac{1}{s-8}+\frac{4}{s-3}}\).
            \\
            \\
            We have that
            \begin{align*}
                \laplace[-1]{\frac{6}{s}-\frac{1}{s-8}+\frac{4}{s-3}}&=\laplace[-1]{\frac{6}{s}}-\laplace[-1]{\frac{1}{s-8}}+\laplace[-1]{\frac{4}{s-3}} \\
                &=6-e^{8t}+4e^{3t}.
            \end{align*}
            
        \end{example}
        \begin{example}{\Difficulty\,\Difficulty\,\,Computing an Inverse Laplace Transform 2}{computinginvlaplace2}

            Compute \(\laplace[-1]{\frac{6s}{s^2+25}+\frac{3}{s^2+25}}\).
            \\
            \\
            We have that
            \begin{align*}
                \laplace[-1]{\frac{6s}{s^2+25}+\frac{3}{s^2+25}}&=\laplace[-1]{\frac{6s}{s^2+25}}+\laplace[-1]{\frac{3}{s^2+25}} \\
                &=6\cos 5t+\frac{3}{5}\sin 5t.
            \end{align*}            
        \end{example}
        \pagebreak
        \begin{example}{\Difficulty\,\Difficulty\,\,Computing an Inverse Laplace Transform 3}{computinginvlaplace3}

            Compute \(\laplace[-1]{\frac{1}{s^3+4s^2+3s}}\).
            \\
            \\
            We have that
            \begin{align*}
                \laplace[-1]{\frac{1}{s^3+4s^2+3s}}&=\laplace[-1]{\frac{1}{s(s^2+4s+3)}} \\
                &=\laplace[-1]{\frac{A}{s}+\frac{B}{s+1}+\frac{C}{s+3}}.
            \end{align*}
            Then, \(1=A(s+1)(s+3)+Bs(s+3)+Cs(s+1)\), so \(A=\frac{1}{3}\), \(B=-\frac{1}{2}\), and \(C=\frac{1}{6}\). Thus,
            \begin{align*}
                \laplace[-1]{\frac{1}{s^3+4s^2+3s}}&=\laplace[-1]{\frac{1}{3}\frac{1}{s}-\frac{1}{2}\frac{1}{s+1}+\frac{1}{6}\frac{1}{s+3}} \\
                &=\frac{1}{3}-\frac{1}{2}e^{-t}+\frac{1}{6}e^{-3t}.
            \end{align*}
            
        \end{example}
        \begin{example}{\Difficulty\,\Difficulty\,\,Computing an Inverse Laplace Transform 4}{computinginvlaplace4}

            Compute \(\laplace[-1]{\frac{-s^5-2s^4+12}{2s^4(s+1)^2}}\).
            \\
            \\
            We have that
            \begin{equation*}
                \laplace[-1]{\frac{-s^5-2s^4+12}{2s^4(s+1)^2}}=\laplace[-1]{\frac{A}{s}+\frac{B}{s^2}+\frac{C}{s^3}+\frac{D}{s^4}+\frac{E}{s+1}+\frac{F}{(s+1)^2}}.
            \end{equation*}
            Then,
            \begin{equation*}
                -s^5-2s^4+12=As^3(s+1)^2+Bs^2(s+1)^2+Cs(s+1)^2+D(s+1)^2+Es^4(s+1)+Fs^4.
            \end{equation*}
            By expanding the right hand side, and matching coefficients, we obtain \(A=-24\), \(B=18\), \(C=-12\), \(D=6\), \(E=\frac{47}{2}\), and \(F=\frac{11}{2}\). Thus,
            \begin{align*}
                \laplace[-1]{\frac{-s^5-2s^4+12}{2s^4(s+1)^2}}&=\laplace[-1]{-\frac{24}{s}+\frac{18}{s^2}-\frac{12}{s^3}+\frac{6}{s^4}+\frac{47}{2}\frac{1}{s+1}+\frac{11}{2}\frac{1}{(s+1)^2}} \\
                &=-24+18t-12t^2+6t^3+\frac{47}{2}e^{-t}+\frac{11}{2}te^{-t}.
            \end{align*}

        \end{example}

\pagebreak

\section{Lecture 23: March 15, 2023}

    \subsection{Solving Differential Equations With the Laplace Transform}
    
        We are now ready to solve linear differential equations using the Laplace Transform. Let
        \begin{equation*}
            F(x,y,y',\ldots,y^{(n)})=f_n(x)y^{(n)}+\cdots+f_1(x)y'+f_0(x)y-Q(x)=0.
        \end{equation*}
        Consider the following steps.
        \begin{enumerate}
            \item Form \(\laplace{F(x,y,y',\ldots,y^{(n)})=f_n(x)y^{(n)}+\cdots+f_1(x)y'+f_0(x)y-Q(x)}=\laplace{0}\).
            \item Solve the resulting algebraic equation for \(\laplace{y(x)}\).
            \item Find \(\laplace[-1]{\laplace{y(x)}}\) in order to find \(y(x)\).
        \end{enumerate}
        \vphantom
        \\
        \\
        Consider the following examples.
        \begin{example}{\Difficulty\,\Difficulty\,\,Using the Laplace Transform as a Solution Method 1}{laplacetransformsol1}
            
            

        \end{example}


\pagebreak

\section{Lecture 24: March 17, 2023}

    \subsection{Generalized Differential Operators: Part I}

        Consider the following definition.
        \begin{definition}{\Stop\,\,Differential Operators}{diffops}
            
            Let \(X\) and \(Y\) be spaces and \(D:X\to Y\). Then, \(D\) is a differential operator if and only if it is linear and satisfies the Leibniz Rule. That is,
            \begin{equation*}
                D(x_1,x_2)=x_1D(x_2)+x_2D(x_1).
            \end{equation*}

        \end{definition}
        \vphantom
        \\
        \\
        For example, for \(C^k(\mathbb{R})\), the space of \(k\)-continuously differentiable functions on \(\mathbb{R}\), the regular differential operator \(D=\frac{\dd}{\dd x}\) satisfies the Leibniz Rule. Note \(\frac{\dd}{\dd x}:C^k(\mathbb{R})\to C^{k-1}(\mathbb{R})\).
        \begin{example}{\Difficulty\,\Difficulty\,\,A Differential Operator}{adiffop}
            
            Given any \(\varphi\in C(\mathbb{R})\), consider \(D=\varphi(x)\frac{\dd}{\dd x}\). Show that \(D\) is a differential operator.
            \\
            \\
            For \(f,g\in C^k(\mathbb{R})\),
            \begin{align*}
                D(c_1f(x)+c_2g(x))&=\varphi(x)\frac{\dd}{\dd x}(c_1f(x)+c_2g(x)) \\
                &=c_1\varphi(x)f(x)+c_2\varphi(x)g(x),
            \end{align*}
            so \(D\) is linear. Then,
            \begin{align*}
                D(f(x),g(x))&=\varphi(x)\frac{\dd}{\dd x}(f(x)g(x)) \\
                &=\varphi(x)f'(x)g(x)+\varphi(x)f(x)g'(x) \\
                &=\varphi(x)f(x)g'(x)+\varphi(x)f'(x)g(x) \\
                &=f(x)D(g(x))+g(x)D(f(x)),
            \end{align*}
            so \(D\) satisfies the Leibniz Rule and satisfies the Leibniz Rule.
            
        \end{example}
        \vphantom
        \\
        \\
        We will now introduce \(2\) common generalized differential operators:
        \begin{equation*}
            \tilde{D}=\varphi_1(x)\frac{\dd}{\dd x}+\cdots+\varphi_n\frac{\dd}{\dd x},\quad \hat{D}=\varphi_1(x)\frac{\dd}{\dd x}+\cdots+\varphi_n\frac{\dd^n}{\dd x^n}.
        \end{equation*}
        \pagebreak
        \vphantom
        \\
        \\
        Consider the following example of computing a differential operator.
        \begin{example}{\Difficulty\,\Difficulty\,\,Computing a Differential Operator 1}{compdiffops1}

            Consider \(D^n=\frac{\dd^n}{\dd x^n}\). Then, find \(((x^2+5)D^2-e^xD^3)(y(x)+\sin x)\).
            \\
            \\
            We have that
            \begin{align*}
                ((x^2+5)D^2-e^xD^3)(y(x)+\sin x)&=((x^2+5)D^2-e^xD^3)y(x)+((x^2+5)D^2-e^xD^3)\sin x \\
                &=(x^2+5)D^2y(x)-e^xD^3y(x)+(x^2+5)D^2\sin x-e^xD^3\sin x \\
                &=(x^2+5)y''(x)-e^xy'''(x)-(x^2+5)\sin x+e^x\cos x.
            \end{align*}

        \end{example}
        \begin{example}{\Difficulty\,\Difficulty\,\,Computing a Differential Operator 2}{compdiffops2}

            Consider \(D^n=\frac{\dd^n}{\dd x^n}\). Then, find \(x^2D(\log(x) D(y(x)))\) and \(\log(x) D(x^2D(y(x)))\).
            \\
            \\
            We have that
            \begin{align*}
                x^2D(\log(x) D(y(x)))&=x^2\left(\log(x) D^2(y(x))+\frac{1}{x}D(y(x))\right) \\
                &=x^2\left(y''(x)\log(x) +y'(x)\frac{1}{x}\right)
            \end{align*}
            and
            \begin{align*}
                \log(x) D(x^2D(y(x)))&=\log(x)\left(x^2D^2(y(x))+2xD(y(x))\right) \\
                &=\log(x)\left(x^2y''(x)+2xy'(x)\right).
            \end{align*}
            Note that \(x^2D(\log(x) D(y(x)))\neq \log(x) D(x^2D(y(x)))\).

        \end{example}
        \begin{example}{\Difficulty\,\Difficulty\,\,Computing a Differential Operator 3}{compdiffops3}

            Consider \(D^n=\frac{\dd^n}{\dd x^n}\). Then, find \((f_n(x)D^n+\cdots+f_0(x)D^0)y(x)-Q(x)\).
            \\
            \\
            We have that
            \begin{align*}
                (f_n(x)D^n+\cdots+f_0(x)D^0)y(x)-Q(x)&=f_n(x)y^{(n)}(x)+\cdots+f_0(x)y(x)-Q(x).
            \end{align*}
            Note that then,
            \begin{equation*}
                (f_n(x)D^n+\cdots+f_0(x)D^0)y(x)-Q(x)=0
            \end{equation*}
            is an \(n\)th order linear differential equation.

        \end{example}

\pagebreak

\section{Lecture 25: March 20, 2023}

    \subsection{Generalized Differential Operators: Part II}

        Consider the following properties of generalized differential operators.
        \begin{theorem}{\Stop\,\,Properties of Generalized Differential Operators}{propsgendiffops}
            
            The following properties hold true for all generalized differential operators.
            \begin{enumerate}
                \item Commutativity of Addition: \(F(D)+G(D)=G(D)+F(D)\).
                \item Associativity of Composition: \(F(D)[G(D)H(D)]=(F(D)G(D))[H(D)]\).
                \item Distributivity I: \(F(D)[G(D)+H(D)]=F(D)[G(D)]+F(D)[H(D)]\).
                \item Distributivity II: \((F(D)+G(D))[H(D)]=F(D)[H(D)]+G(D)[H(D)]\).
                \item Additive Identity \(0\cdot\frac{\dd}{\dd x}=0\).
                \item Compositional Identity: \(1\cdot D^0=1\).
            \end{enumerate}

        \end{theorem}
        \vphantom
        \\
        \\
        Consider the following definitions and theorems.
        \begin{definition}{\Stop\,\,Polynomial Differential Operators}{polydiffops}

            We define \(P_n(D)=a_nD^n+\cdots+a_0\), where \(a_0,\ldots,a_n\in\mathbb{C}\), as a polynomial differential operator of order \(n\).
            
        \end{definition}
        \begin{theorem}{\Stop\,\,Polynomial Operators Satisfy Compositional Commutativity}{polyopscompcom}

            For polynomial differential operators \(P_m(D)\) and \(P_n(D)\),
            \begin{equation*}
                P_m(D)[P_n(D)]=P_n(D)[P_m(D)]=P_{m+n}(D).
            \end{equation*}
            Moreover, if \(P_m(D)=a_nD^n+\cdots+a_0\) and \(P_n(D)=b_nD^n+\cdots+b_0\), \(P_{m+n}(D)\) has coefficients obtained through ordinary polynomial multiplication.
            
        \end{theorem}
        \begin{theorem}{\Stop\,\,Factorization of Polynomial Operators}{factorpoly}

            If \(a_nm^n+\cdots+a_0=(m-m_1)\cdots(m-m_n)\) for \(m_1,\ldots,m_n\in\mathbb{C}\), 
            \begin{equation*}
                P_n(D)=(D-m_1)\cdots(D-m_n).
            \end{equation*}
            
        \end{theorem}
        \pagebreak
        \vphantom
        \\
        \\
        Consider the following example.
        \begin{example}{\Difficulty\,\Difficulty\,\,An Example Involving Theorem \ref{thm:factorpoly}}{factorpolyexa}

            Show that
            \begin{equation*}
                P_3(D)=D^3-3D^2+2D=D[(D-2)[D-1]].
            \end{equation*}
            By Theorem \ref{thm:factorpoly}, this is true since \(m^3-3m^2+2m=m(m-2)(m-1)\). But, to explicitly show this, consider
            \begin{align*}
                (D^3-3D^2+2D)[y(x)]=y'''(x)-3y''(x)+2y'(x),
            \end{align*}
            and then
            \begin{align*}
                D[(D-2)[D-1]][y(x)]&=D[(D-2)[y'(x)-y(x)]] \\
                &=D[y''(x)-3y'(x)+2y(x)] \\
                &=y'''(x)-3y''(x)+2y'(x),
            \end{align*}
            as desired.
            
        \end{example}
        \vphantom
        \\
        \\
        Since \(P_n(D)[y]=0\) defines a linear ordinary differential equation with constant coefficients, the characteristic polynomial is precisely obtained by replacing \(D\) with \(m\). Note that factorization is still sometimes possible for some general differential operator \(F(D)\). Take \(x^2y''+2xy'=0\) as an example. Then, \(x^2D^2+2xD=D[x^2D]\).

\pagebreak

\section{Lecture 26, April 3, 2023}

    \subsection{The Method of Operators}

        Suppose \(F(x,y,\ldots,y^{(n)})=0\) is linear and can be ``factored'' in terms of generalized differential operators such that we have
        \begin{equation*}
            F_1(D)\cdots F_n(D)+Q(x)=0.
        \end{equation*}
        Then, \(F(x,y,\ldots,y^{(n)})=0\) can be solved recursively by solving \(n\) first order equations in sequence. Consider the following examples.
        \begin{example}{\Difficulty\,\Difficulty\,\,The Method of Operators 1}{methodops1}

            Solve \(F(x,y,y',y'')=y''-y'-2e^x=0\) using the Method of Operators.
            \\
            \\
            We have that \(y_c(x)=c_1+c_2e^x\). Note that \(y''-y'=(D^2-D)y(x)=D(D-1)(y(x))\). Therefore,
            \begin{equation*}
                F(x,y,y',y'')=D(D-1)(y_p(x))-2e^x=0.
            \end{equation*}
            Let \(v(x)=(D-1)(y_p(x))\). Then, we have the two first order equations
            \begin{equation*}
                Dv(x)-2e^x=0,\quad (D-1)y_p(x)-v(x)=0.
            \end{equation*}
            The first equation gives us \(v'(x)=2e^x\), so \(v(x)=2e^x+c\). Since we are only looking for a particular solution, let \(c=0\) so \(v(x)=2e^x\). The second equation gives us \(y_p'(x)-y_p(x)-2e^x=0\). Thus, \(y_p(x)=\frac{2x}{e^{-x}}=2xe^x\). Thus, we have that
            \begin{equation*}
                y(x)=c_1+c_2e^x+2xe^x
            \end{equation*}
            on \(\mathbb{R}\).
        \end{example}
        \begin{example}{\Difficulty\,\Difficulty\,\,The Method of Operators 2}{methodops2}

            Solve \(F(x,y,y',y'')=x^2y''-2xy'+5=0\) using the Method of Operators.
            \\
            \\
            We have that \((x^2D^2-2xD)(y(x))+5=(x^2D-2x)(D(y(x)))+5\). Then,
            \begin{equation*}
                F(x,y,y',y'')=(x^2D-2x)(D(y(x)))+5=0.
            \end{equation*}
            Let \(v(x)=D(y(x))\). Then we have the two first order equations
            \begin{equation*}
                (x^2D-2x)v(x)+5=0,\quad Dy(x)-v(x)=0.
            \end{equation*}
            The first equation gives us \(x^2v'(x)-2xv(x)+5=0\). With \(x\neq 0\), we have \(v'(x)-\frac{2}{x}v(x)+\frac{5}{x^2}=0\). Thus,
            \(v(x)=x^2\left(\int -\frac{5}{x^4}\dd x\right)+c_1=\frac{5}{3x}+c_1x^2\). The second equation gives us \(y'(x)-\frac{5}{3x}-c_1x^2=0\). Then,
            \begin{equation*}
                y(x)=\int \frac{5}{3x}\dd x+\int c_1x^2\dd x=\frac{5}{3}\log|x|+c_2+\tilde{c}_1x^3
            \end{equation*}
            on \(\{x\in\mathbb{R}:x\neq0\}\).

        \end{example}