\section{Lecture 31: April 14, 2023}

    \subsection{A Review of Power Series}

        We seek to solve differential equations using power series; this is due to the fact that, excluding constant coefficient equations, we have seen enormous difficulty finding closed form solutions to ``nice''-looking equations. For example,
        consider the following.
        \begin{enumerate}
            \item \(x^2y''-2xy_2y=0\).
            \item \(y''-xy=0\).
            \item \(xy'''-y''=0\).
            \item \(xy'''-y'=0\).
            \item \(xy''+y'-y=0\).
        \end{enumerate}
        \pagebreak
        \vphantom
        \\
        \\
        Note that some of the above equations are not too difficult, but distinguishing these from the equations that are is very difficult; sometimes, we may not be able to compute antiderivatives at all. We now will consider power series. Consider the following definition.
        \begin{definition}{\Stop\,\,Power Series}{powerseries}

            Let \(a_i\in\mathbb{C}\). The power series centered at \(x_0\in\mathbb{R}\) is given by
            \begin{equation*}
                P(x-x_0)=\sum_{k=0}^\infty a_k(x-x_0)^k
            \end{equation*}
            where \(P(x-x_0)\) converges at \(x\) if and only if
            \begin{equation*}
                \lim_{n\to\infty}\sum_{k=0}^n a_k(x-x_0)^k
            \end{equation*}
            exists and is finite.

        \end{definition}
        \begin{definition}{\Stop\,\,Radius of Convergence}{radiusofconvergence}

            The power series \(P(x-x_0)\) has a radius of convergence \(R\in\mathbb{R}^+\) if it converges for all \(x\in\{x\in\mathbb{R}:x_0-R<x<x_0+R\}\). Note that \(P(x-x_0)\) may or may not converge at \(x=x_0\pm R\).

        \end{definition}
        \begin{definition}{\Stop\,\,Valid Power Series Representation}{validpowerseriesrep}
            
            We define \(f(x)\) has a valid power series representation \(P(x-x_0)\) on some interval \(I\) if for all \(x\in I\), \(P(x-x_0)\) converges to \(f(x)\). That is,
            \begin{equation*}
                \lim_{n\to\infty}\left|f(x)-\sum_{k=0}^n a_k(x-x_0)^k\right|=0.
            \end{equation*}
            If the above equation is true, we write \(f(x)\simeq P(x-x_0)\).

        \end{definition}
        \begin{definition}{\Stop\,\,Absolute Convergence}{absoluteconvergence}
            
            If \(\sum_{k=0}^\infty |a_k|(x-x_0)^k\) is convergent, \(\sum_{k=0}^\infty a_k(x-x_0)^k\) converges absolutely.

        \end{definition}
        \begin{theorem}{\Stop\,\,A Useful Lemma for Absolute Convergence}{lemmaabsconv}

            If \(P(x-x_0)\) is absolutely convergent, it is convergent. Moreover, \(P(x-x_0)\) is convergent regardless of the ordering of \(a_k\).
            
        \end{theorem}
        
\section{Lecture 32: April 17, 2023}

    \subsection{Existence and Uniqueness for Power Series Solutions}

        Consider the following theorems.
        \begin{theorem}{\Stop\,\,Uniqueness of Valid Power Series Expansions}{uniquenessforvalidpowerseries}
            
            If \(f(x)\) possesses a valid expansion \(P(x-x_0)\) for some \(x\in\{x\in\mathbb{R}:x_0-R<x<x_0+R\}\), this expansion is unique. Moreover, the coefficient terms are
            \begin{equation*}
                a_k=\frac{f^{(k)}(x_0)}{k!}.
            \end{equation*}
            This is just the usual Taylor expansion.
        \end{theorem}
        \begin{theorem}{\Stop\,\,Taylor Series Remainder}{taylorremainder}

            Recall that by Definition \ref{def:validpowerseriesrep}, \(f(x)\) has a valid power series representation \(P(x-x_0)\) if it converges to \(f(x)\); that is, the remainder must approach zero as \(n\) approaches infinity. We can give the remainder as
            \begin{equation*}
                R_n(x)=\frac{f^{(n+1)}(\xi)(x-x_0)^{n+1}}{(n+1)!}
            \end{equation*}
            where \(\xi\) is in the interval between \(x\) and \(x_0\).
            
        \end{theorem}
        \vphantom
        \\
        \\
        Consider the following example of finding a Taylor series.
        \begin{example}{\Difficulty\,\Difficulty\,\,Finding a Taylor Series}{findtaylor}
            
            Find the formal Taylor series expansion for \(f(x)=\frac{1}{1-x}\) centered at \(x_0=0\).
            \\
            \\
            Note \(f(x_0)=\frac{1}{1-x},f'(x_0)=\frac{1}{(1-x)^2}, f''(x_0)=\frac{2}{(1-x)^3},\ldots,f^{(k)}(x_0)=\frac{k!}{(1-x)^{k+1}}\). At \(x_0=0\), \(f^{(k)}(0)=k!\). Then, the coefficients are given by \(a_k=\frac{k!}{k!}=1\), and so we have the expected expansion
            \begin{equation*}
                f(x)\simeq\sum_{k=0}^\infty x^k.
            \end{equation*}

        \end{example}
        \vphantom
        \\
        \\
        The following notion of analytic functions is important.
        \begin{definition}{\Stop\,\,Analytic Functions}{analyticfunctions}
            
            A function \(f:\mathbb{R}\to\mathbb{R}\) is analytic at \(x\) if and only if there exists a valid power series expansion on some neighborhood \((x_0-R,x_0+R)\). Given some interval \(I\), if \(f\) is analytic at every point \(x\in I\), it is analytic on \(I\).

        \end{definition}
        \pagebreak
        \vphantom
        \\
        \\
        Consider the following theorem.
        \begin{theorem}{\Stop\,\,Analytic Implies Existence of Power Series General Solution}{analyticimppowergensol}
            
            Let \(F(x,y,\ldots,y^{(n)})=f_n(x)y^{(n)}+\cdots+f_1(x)y'+f_0(x)y-Q(x)=0\) be an \(n\)th order linear equation with the additional restrictions that \(f_i\), \(1\leq i\leq n\) and \(Q(x)\) are analytic on some common interval \(I\). Under these stipulations, the general solution \(y(x)\) possesses a power series solution valid on \(I\).

        \end{theorem}
        \vphantom
        \\
        \\
        This leads us to a question. What is the relationship between \(y'(x)\) and \(y(x)\simeq P(x-x_0)\)?
        \\
        \\
        If \(P(x-x_0)=\sum_{k=0}^\infty a_k(x-x_0)^k=\sum_{k=0}^\infty p_k(x)\) and \(P'(x-x_0)=\sum_{k=0}^\infty p_k'(x)\), then, if \(P'(x-x_0)\) converges uniformly to some function,
        \begin{equation*}
            y'(x)\simeq P'(x-x_0).
        \end{equation*}
        Note \(p_i\) is a polynomial. Now, consider the following definitions.
        \begin{definition}{\Stop\,\,Singularities}{singularities}

            Let \(F(x,y,\ldots,y^{(n)})=0\) be an \(n\)th order homogeneous linear equation. A point \(x_0\) is ordinary if and only if all \(f_i\), \(1\leq i \leq n\), are analytic at \(x_0\). Then, \(x_0\) is singular if and only if at least one \(f_i\) is not analytic at \(x_0\).
            \\
            \\
            A singularity is regular if and only if, after placing the equation into standard form, all \(g_i\) are analytic at \(x_0\). Note
            \begin{equation*}
                g_i(x)=f_i(x)(x-x_0)^{n-i}.
            \end{equation*}
            A singularity is irregular if and only if it is not regular.
        \end{definition}

\section{Lecture 33: April 19, 2023}

\pagebreak

\section{Lecture 34: April 21, 2023}

    \subsection{Frobenius Equations and Series Solutions: Part I}

    Recall that \(g_0(x)=x-1\) and \(g_1(x)=x\) are polynomials, so both functions are analytic. Thus, \(x_0=1\) is a regular singularity.
    \\
    \\
    If \(x_0=a\) is an ordinary point, \(y(x)\) will jave a power series solution \(y(x)\simeq P(x-x_0)\) valid on the common interval of analyticity. In the case \(x_0\) is a regular sngularity, \(y(x)\) need not have a power series expansion. We now define Frobenius equations and state an accompanying theorem.
    \begin{definition}{\Stop\,\,Frobenius Equations}{frobeniusequation}

        If \(F(x,y\ldots,y^{(n)})=0\) is in standard form and \(x_0\) is a regular singularity, the associated Frobenius differential equation is
        \begin{equation*}
            (x-x_0)^nF(x,y,\ldots,y^{(n)})=0.
        \end{equation*}
        
    \end{definition}
    \begin{theorem}{\Stop\,\,Solution to the Frobenius Equation}{frobeniusseries}

        The Frobenius equation has the same solution as \(F\), given by 
        \begin{equation*}
            y(x)\simeq \sum_{k=0}^\infty a_{k,m}(x-x_0)^{k+m}
        \end{equation*}
        with \(a_{0,m}\neq0\) and \(m\in\mathbb{C}\). This Frobenius series \(P(x-x_0,m)\) is valid on the common interval of analyticity of the coefficient functions of the Frobenius equation.
        
    \end{theorem}
    \pagebreak
    \vphantom
    \\
    \\
    Consider the following example.
    \begin{example}{\Difficulty\,\Difficulty\,\,Finding Power Series Coefficients 1}{findpowerseriescoeffs1}

        Consider \(F(x,y,y',y'')=(1-x)y''+xy'-y=0\).
        \\
        \\
        We have proven \(x_0=0\) is an ordinary point. The interval of analyticity, we have shown to be \(I=\{x\in\mathbb{R}:-1<x<1\}\). We are expecting a power series solution \(y(x)\simeq P(x-x_0)\) valid on \(I\). Since \(x_0=0\) is ordinary, we will use the easier non-standard form. We now assume
        \begin{equation*}
            y(x)=\sum_{k=0}^\infty a_k(x-0)^k=\sum_{k=0}^\infty a_kx^k,
        \end{equation*}
        so
        \begin{equation*}
            y'(x)=\sum_{k=0}^\infty ka_kx^{k-1},\quad y''(x)=\sum_{k=0}^\infty k(k-1)a_kx^{k-2}.
        \end{equation*}
        We may now substitute this into \(F\) to obtain
        \begin{align*}
            (1-x)\sum_{k=0}^\infty k(k-1)a_kx^{k-2}+x\sum_{k=0}^\infty ka_kx^{k-1}-\sum_{k=0}^\infty a_kx^k&=0 \\
            \sum_{k=0}^\infty k(k-1)a_kx^{k-2}-\sum_{k=0}^\infty k(k-1)a_kx^{k-1}+\sum_{k=0}^\infty ka_kx^k-\sum_{k=0}^\infty a_kx^k&= \\
            \sum_{k=2}^\infty k(k-1)a_kx^{k-2}-\sum_{k=2}^\infty k(k-1)a_kx^{k-1}+\sum_{k=0}^\infty (k-1)a_kx^k&= \\
            \sum_{k=0}^\infty (k+2)(k+1)a_{k+2}x^k-\sum_{k=0}^\infty k(k+1)a_{k+1}x^k+\sum_{k=0}^\infty (k-1)a_kx^k&=
        \end{align*}
        Then,
        \begin{equation*}
            \sum_{k=0}^\infty ((k-1)a_k-k(k+1)a_{k+1}+(k+2)(k+1)a_{k+2})x^k=0\simeq\sum_{k=0}^\infty 0x^k.
        \end{equation*}
        We now have the equation \((k-1)a_k-k(k+1)a_{k+1}+(k+2)(k+1)a_{k+2}=0\), a recurrence relation. We test small values of \(k\) to see if we can find a pattern. We wish to identify all coefficients in terms of \(a_0\).
        \begin{itemize}
            \item \(k=0\): \(a_0-0+2\cdot1\cdot a_2=0\) so \(2a_2=a_0\).
            \item \(k=1\): \(-2\cdot1\cdot a_2+6a_3=0\) so \(a_3=\frac{1}{3}a_2=\frac{1}{6}a_0\).
            \item \(k=2\): \(a_2-6a_3+4\cdot3 a_4=0\) so \(a_4=\frac{1}{2}a_3-\frac{1}{12}a_2=\frac{1}{24}a_0\).
            \item \(\cdots\): \(\cdots\) so \(\cdots\).
            \item \(k=n\): \(\cdots\) so \(a_{n+2}=\frac{1}{(n+2)!}a_0\).
        \end{itemize}
        \vphantom
        \\
        \\
        Note that \(a_1\) is arbitrary and is not dependent on any other \(a_k\). Thus, on \((-1,1)\),
        \begin{equation*}
            y(x)\simeq \sum_{k=0}^\infty a_kx^k=a_0+a_1x+\sum_{k=2}^\infty \frac{a_k}{k!}x^k=a_0+a_0x+(a_1-a_0)x+\sum_{k=2}^\infty \frac{a_0}{k!}x^k=(a_1-a_0)x+\sum_{k=0}^\infty \frac{a_0}{k!}a^k=\tilde{a_1}x+a_0e^x.
        \end{equation*}
    \end{example}

\pagebreak

\section{Lecture 35: April 24, 2023}

    \subsection{Frobenius Equations and Series Solutions: Part II}

        We now proceed by working out more examples.
        \begin{example}{\Difficulty\,\Difficulty\,\,Finding Power Series Coefficients 2}{findpowerseriescoeffs2}
            
            Consider \(F(x,y,y',y'')=x^2y''-3xy'+2y=0:x_0=0\).
            \\
            \\
            We first determine the nature of \(x_0=0\) by first writing \(F\) in standard form. We obtain
            \begin{equation*}
                y''-\frac{3}{x}y'+\frac{2}{x^2}y=0.
            \end{equation*} 
            Note that \(f_0(x)=\frac{2}{x^2}\). Since \(f_0(x)\) is not defined at \(x_0=0\), it cannot ba analytic at \(x_0\). Thus, \(x_0\) is a singular point by Definition \ref{def:singularities}. Then, we construct \(g_0(x)=f_0(x)(x-x_0)^2=\frac{2}{x^2}x^2=2\), \(g_1(x)=f_1(x)(x-x_0)^2=-\frac{3}{x}x=-3\). Since both \(g_0\) and \(g_1\) are constant, they are analytic everythere. Thus, \(F\) has a Frobenius solution valid on \(\mathbb{R}\), except perhaps at \(x=0\). We have a regular singularity at \(x_0=0\). Now, we consider the Frobenius equation 
            \begin{align*}
                F(x,y,y',y'')_{\text{frob}}&=(x-x_0)^2F(x,y,y',y'') \\
                &=x^2\left(y''-\frac{3}{x}y'+\frac{2}{x^2}\right)=0 \\
                &=x^2y''-3xy'+2y=0.
            \end{align*}
            Note that in this case, \(F_{\text{frob}}\) is the same as \(F\), but this is not true in general. Now, we suppose
            \begin{equation*}
                y(x)=\sum_{k=0}^\infty a_{k,m}(x-x_0)^{k+m}=\sum_{k=0}^\infty a_{k,m}x^{k+m},
            \end{equation*}
            where \(a_{0,m}\neq0\) and \(m\in\mathbb{C}\). Then,
            \begin{equation*}
                y'(x)=\sum_{k=0}^\infty a_{k,m}(k+m)a_{k,m}x^{k+m-1}
            \end{equation*}
            and
            \begin{equation*}
                y''(x)=\sum_{k=0}^\infty a_{k,m}(k+m)(k+m-1)x^{k+m-2}.
            \end{equation*}
            Substituting into \(F_{\text{frob}}\), we have
            \begin{align*}
                x^2\sum_{k=0}^\infty a_{k,m}(k+m)(k+m-1)x^{k+m-2}-3x\sum_{k=0}^\infty a_{k,m}(k+m)a_{k,m}x^{k+m-1}+2\sum_{k=0}^\infty a_{k,m}(x-x_0)^{k+m}=0.
            \end{align*}
            We can simplify the above to produce
            \begin{align*}
                \sum_{k=0}^\infty a_{k,m}(k+m)(k+m-1)x^{k+m}-3\sum_{k=0}^\infty a_{k,m}(k+m)a_{k,m}x^{k+m}+2\sum_{k=0}^\infty a_{k,m}(x-x_0)^{k+m}=0.
            \end{align*}
            Now, we can combine terms to obtain
            \begin{equation*}
                \sum_{k=0}^\infty ((k+m)(k+m-1)-3(k+m)+2)a_{k,m}x^{k+m}=0\simeq \sum_{k=0}^\infty 0x^{k+m}.
            \end{equation*}
            Thus, \(((k+m)(k+m-1)-3(k+m)+2)a_{k,m}=0\). Since \(a_{0,m}\neq0\), evaluating at \(k=0\), we get
            \begin{equation*}
                (m(m-1)-3m+2)a_{0,m}=0,
            \end{equation*}
            meaning \(m^2-4m+2=0\), which is true for \(m_1=2+\sqrt{2}\) and \(m_2=2-\sqrt{2}\). For \(m_1\), and \(k\geq 1\), we have
            \begin{equation*}
                ((k+2+\sqrt{2})(k+1+\sqrt{2})-3(k+2+\sqrt{2})+2)a_{k,m_1}=0
            \end{equation*}
            which has no integer roots for \(k\). Thus, \(a_{k,m_1}=0\) for all \(k\geq 1\). For \(m_2\), and \(k\geq 1\), we have
            \begin{equation*}
                ((k+2-\sqrt{2})(k+1-\sqrt{2})-3(k+2-\sqrt{2})+2)a_{k,m_2}=0
            \end{equation*}
            which has no integer roots for \(k\). Thus, \(a_{k,m_2}=0\) for all \(k\geq 1\). Thus, \(m_1\), we obtain the series
            \begin{equation*}
                y_1(x)\simeq a_{0,m_1}x^{0+m_1}+\sum_{k=1}^\infty 0x^{k+m},
            \end{equation*}
            so
            \begin{equation*}
                y_1(x)=a_{0,m_1}x^{2+\sqrt{2}}.
            \end{equation*}
            Similarly, for \(m_2\), we have
            \begin{equation*}
                y_2(x)=a_{0,m_2}x^{2-\sqrt{2}}.
            \end{equation*}
            Since both \(y_1\) and \(y_2\) are both linearly independent \(1\)-parameter families of solutions, the existence and uniqueness for linear equations gives
            \begin{equation*}
                y(x)=a_{0,m_1}x^{2+\sqrt{2}}+a_{0,m_2}x^{2-\sqrt{2}}
            \end{equation*}
            where \(a_{0,m_i}\in\mathbb{C}\). We must check this solution at \(x=0\). This is valid since \(y(x)\) is well-defined at \(x=0\) and so was the original equation \(F\).
        \end{example}

\pagebreak

\section{Lecture 35: April 24, 2023}

    \subsection{Frobenius Equations and Series Solutions: Part III}

        In the best case, where the roots of the indical polynomial differ by more than an integer, each root \(m_i\) gives rise to at least one \(1\)-parameter family of solutions. Sometimes, with repeated roots, we can get a \(k\)-parameter family from a single root for \(k\geq 2\).
        \\
        \\
        However, when \(m_i-m_j\in\mathbb{Z}-\{0\}\), it is possible that a ``logarithmic'' family exists not obtainable in the standard way. To illustrate, consider the following example.
        \begin{example}{\Difficulty\,\Difficulty\,\,Finding Power Series Coefficients 3}{findpowerseriescoeffs3}
            
            Consider \(F(x,y,y',y'',y''')=x^2y'''-2xy''-4y'=0:x_0=0\).
            \\
            \\
            We first deterine the nature of \(x_0=0\) by first writing \(F\) in standard form. We obtain
            \begin{equation*}
                y'''-\frac{2}{x}y''-\frac{4}{x^2}y'=0.
            \end{equation*}
            Note that \(f_1(x)=-\frac{4}{x^2}\). Since \(f_1(x)\) is not defined at \(x_0=0\), it cannot be analytic at \(x_0\). Thus \(x_0\) is a singular point by Definition \ref{def:singularities}. Then, we construct \(g_0(x)=0\), \(g_1(x)=-\frac{4}{x^2}x^2=-4\), \(g_2=(x)=-\frac{2}{x}x=-2\). Since all \(g_i\) are constant, they are analytic everywhere. Thus, \(F\) has a Fronbenius solution valid on \(\mathbb{R}\), except perhaps at \(x=0\). We have a regular singularity at \(x_0=0\). Now, we consider the Frobenius equation
            \begin{align*}
                F(x,y,y',y'',y''')_{\text{frob}}&=(x-x_0)^3F(x,y,y',y'',y''') \\
                &=x^3y'''-2x^2y''-4xy'=0.
            \end{align*}
            Now, we suppose
            \begin{equation*}
                y(x)=\sum_{k=0}^\infty a_{k,m}x^{k+m}
            \end{equation*}
            where \(a_{0,m}\neq0\) and \(m\in\mathbb{C}\). Then,
            \begin{equation*}
                y'(x)=\sum_{k=0}^\infty a_{k,m}(k+m)x^{k+m-1},\quad y''(x)=\sum_{k=0}^\infty a_{k,m}(k+m)(k+m-1)x^{k+m-2},
            \end{equation*}
            and
            \begin{equation*}
                y'''(x)=\sum_{k=0}^\infty a_{k,m}(k+m)(k+m-1)(k+m-2)x^{k+m-3}.
            \end{equation*}
            Substituting into \(F_{\text{frob}}\), we have
            \begin{align*}
                x^3\sum_{k=0}^\infty a_{k,m}(k+m)(k+m-1)(k+m-2)x^{k+m-3}&-2x^2\sum_{k=0}^\infty a_{k,m}(k+m)(k+m-1)x^{k+m-2}\\
                &-4x\sum_{k=0}^\infty a_{k,m}(k+m)x^{k+m-1}=0
            \end{align*}
            or, equivalently,
            \begin{align*}
                \sum_{k=0}^\infty a_{k,m}(k+m)(k+m-1)(k+m-2)x^{k+m}&-2\sum_{k=0}^\infty a_{k,m}(k+m)(k+m-1)x^{k+m}\\
                &-4\sum_{k=0}^\infty a_{k,m}(k+m)x^{k+m}=0.
            \end{align*}
            Now we can combine terms to obtain
            \begin{equation*}
                \sum_{k=0}^\infty ((k+m)(k+m-1)(k+m-2)-2(k+m)(k+m-1)-4(k+m))a_{k,m}x^{k+m}=0\simeq \sum_{k=0}^\infty 0x^{k+m}.
            \end{equation*}
            Thus, \(((k+m)(k+m-1)(k+m-2)-2(k+m)(k+m-1)-4(k+m))a_{k,m}=0\). Since \(a_{0,m}\neq0\), evaluating at \(k=0\), we get
            \begin{equation*}
                (m(m-1)(m-2)-2m(m-1)-4m)a_{0,m}=0,
            \end{equation*}
            meaning \(m^2(m-5)=0\), which is true for \(m_1=0\) and \(m_2=5\). For \(m_1\), and \(k\geq 1\), we have
            \begin{equation*}
                (k(k-1)(k-2)-2k(k-1)-4k)a_{k,m_1}=k^2(k-5)a_{k,m_1}=0.
            \end{equation*}
            For \(k=5\), \(a_{5,m_1}\in\mathbb{C}\), but for \(k\neq5\), \(a_{k,m_1}=0\). Thus, \(y_1(x)\simeq a_{0,m_1}x^{0+m_1}+\sum_{k=1}^\infty a_{k,m_1}x^{k+m_1}=a_{0,m_1}x^0+a_{5,m_1}x^5=a_{0,m_1}+a_{5,m_1}x^5\). For \(m_2\), and \(k\geq 1\), we have
            \begin{equation*}
                ((k+5)(k+4)(k+3)-2(k+5)(k+4)-4(k+5))a_{k,m_2}=k(k+5)^2.
            \end{equation*}
            Thus, \(y_2(x)=a_{0,m_2}x^{0+m_2}+\sum_{k=1}^\infty a_{k,m_2}x^{k+m_2}=a_{0,m_2}x^5\). This is information we had previously. We must find some third linearly independent \(1\)-parameter family solutions, since \(F\) is of order \(3\). A fact is \(y_3(x)=\log x\).
        \end{example}