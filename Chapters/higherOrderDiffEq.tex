\section{Lecture 11: February 13, 2023}

    \subsection{Linear Ordinary Differential Equations}

        Consider the following definition.
        \begin{definition}{\Stop\,\,Linear Ordinary Differential Equations}{nthorderlineardiffeq}

            An \(n\)th order linear ordinary differential equation, with respect to \(I\), is
            \begin{equation*}
                F(x,y,y',\ldots,y^{(n)})=f_n(x)y^{(n)}+\cdots+f_1(x)y'+f_0(x)y+Q(x)=0
            \end{equation*}
            where all \(f_i\), \(0\leq i\leq n\), and \(Q\) are continuous on some common interval \(I\) and \(f_n(x)\neq0\) for all \(x\in I\).
            
        \end{definition}
        \vphantom
        \\
        \\
        Note that every first order linear differential equation can be written as 
        \begin{equation*}
            F(x,y,y')=f_1(x)y'+f_0(x)y+Q(x)=y'+\tilde{P}(x)y+\tilde{Q}(x)=0.
        \end{equation*}
        and has the integrating factor \(\mu(x,y)=\mu(x)=e^{\int \tilde{P}(x)\dd x}\).
        \\
        \\
        Consider the following non-example of a linear ordinary differential equation.
        \begin{example}{\Difficulty\,\Difficulty\,\,A Non-Example of a Linear Ordinary Differential Equation}{linorddiffeq}

            Consider \(F(x,y,y',y'',y'')=y'''+y''\log(x-1)+y\arccos(x)-\log(x-2)=0\).
            \\
            \\
            The differential equation \(F(x,y,y',y'',y''')\) is not a linear ordinary differential equation since \(\log(x-1)\), \(\arccos(x)\), and \(\log(x-2)\) are not continuous on any common interval.
            
        \end{example}

\section{Lecture 12: February 20, 2023}

    \subsection{Existence and Uniqueness of Solutions for Linear Differential Equations}

        Consider the following definition.
        \begin{theorem}{\Stop\,\,Linear Independence}{linindep}

            Let \(S=\{f_i:X\to\mathbb{C}:1\leq i\leq n\}\) be a set of continuous complex-valued functions defined on \(X\subseteq\mathbb{R}\). Then, \(S\) is linearly independent if and only if, for all \(x\in X\), 
            \begin{equation*}
                c_1f_1(x)+\cdots+c_nf_n(x)=0
            \end{equation*}
            with \(c_1,\ldots,c_n\in\mathbb{C}\), is satisfied only if \(c_1=\cdots=c_n=0\). Note that \(S\) is linearly dependent if and only if \(S\) is not linearly independent.
            
        \end{theorem}
        \vphantom
        \\
        \\
        Consider the following theorems.
        \begin{theorem}{\Stop\,\,Existence of Solutions to \(n\)th Order Linear Differential Equations (I)}{existence1}
            
            Let \(F(x,y,\ldots,y^{(n)})\) be an \(n\)th order linear differential equation with respect to \(I\). Then, there exists \(y(x)\) such that
            \begin{equation*}
                y(x_0)=y_0,\quad y'(x_0)=y_1,\quad \ldots,\quad y^{(n-1)}(x_0)=y_{n-1}
            \end{equation*}
            are initial conditions on \(F\) satisfied for any \(x_0\in I\).

        \end{theorem}
        \begin{theorem}{\Stop\,\,Existence of Solutions to \(n\)th Order Linear Differential Equations (II)}{existence2}
            
            Let \(F(x,y,\ldots,y^{(n)})\) be an \(n\)th order linear differential equation with respect to \(I\). Then, there exist \(n\) \(1\)-parameter families of solutions \(c_iy_i(x)\) for \(c_i\in\mathbb{C}\) and some particular solution \(y_i(x)\) valid on \(I\). Moreover, the set \(S=\{y_1(x),\ldots,y_n(x)\}\) is linearly independent and \(c_1y_1(x)+\cdots+c_ny_n(x)\) is an \(n\)-parameter family of solutions valid on \(I\).

        \end{theorem}
        \begin{theorem}{\Stop\,\,Uniqueness of Solutions to \(n\)th Order Linear Differential Equations}{uniqueness}
            
            Let \(F(x,y,\ldots,y^{(n)})\) be an \(n\)th order linear differential equation with respect to \(I\). Then, \(F\) has a unique solution valid on \(I\) satisfying \(n\) arbitrary initial conditions.

        \end{theorem}
        \vphantom
        \\
        \\
        Therefore, by Theorem \ref{thm:uniqueness}, the \(n\)-parameter family of solutions specified in Theorem \ref{thm:existence2} is the \(y(x)\) specified in Theorem \ref{thm:existence1}; note that \(y(x)\) is a general solution.
        \pagebreak
        \vphantom
        \\
        \\
        \begin{definition}{\Stop\,\,Wronskians}{wronskians}

            Let \(S=\{f_i:X\to\mathbb{C}:1\leq i\leq n\}\) be a set of continuous complex-valued functions defined on \(X\subseteq\mathbb{R}\). Suppose that each \(f_i\in S\), \(1\leq i\leq n\) us \((n-1)\)-times continuously differentiable. Then, the Wronskian of \(S\) is given by
            \begin{equation*}
                W(S)(x)=\det\begin{bmatrix}
                    f_1(x) & \cdots & f_n(x) \\
                    \vdots & \ddots & \vdots \\
                    f_1^{(n-1)}(x) & \cdots & f_n^{(n-1)}(x).
                \end{bmatrix}
            \end{equation*}
            
        \end{definition}
        \begin{theorem}{\Stop\,\,Wronskians Determine Linear Independence}{wronskianslinindep}

            If \(S\) is linearly dependent, \(W(S)(x)=0\) for all \(x\in X\). If \(W(S)(x)\neq0\) for some \(x\in X\), \(S\) is linearly independent on \(X\). 
            
        \end{theorem}
        \vphantom
        \\
        \\
        Consider the following example.
        \begin{example}{Showing Linear Independence With the Wronskian}{showlinindepwronksian}

            Show that \(\{x,e^{2x},e^{3x}\}\) are linearly independent on \(\{x\in\mathbb{R}:0\leq x\leq 1\}\).
            \\
            \\
            Note that all elements of \(S\) are \((n-1)\)-times continuously differentiable. Then,
            \begin{align*}
                W(S)(x)&=\det\begin{bmatrix}
                    x & e^{2x} & e^{3x} \\
                    1 & 2e^{2x} & 3e^{3x} \\
                    0 & 4e^{2x} & 9e^{3x}
                \end{bmatrix} \\
                &=-6xe^{5x}+5e^{5x}.
            \end{align*}
            Note that \(W(S)(0)=5\neq0\), so \(S\) is linearly independent.
            
        \end{example}
        \begin{theorem}{\Stop\,\,Wronksians are Necessary and Sufficient in Some Cases}{wronskianslinindep2}

            If \(S=\{y_i:X\to\mathbb{C}\}\) and all \(y_i\), \(1\leq i\leq n\), are solutions to the same \(n\)th order linear differential equation, \(W(S)(x)=0\) on \(X\) if and only if \(S\) is linearly dependent.
            
        \end{theorem}

\pagebreak

\section{Lecture 13: Feburary 22, 2023}

    \subsection{Linear Differential Equations With Constant Coefficients}

        We will now study linear differential equations with constant coefficients. Consider the following definitions.
        \begin{definition}{\Stop\,\,Linear Differential Equations With Constant Coefficients}{constcoeff}

            An \(n\)th order linear homogeneous differential equation, with respect to \(I\), is 
            \begin{equation*}
                F(x,y,y',\ldots,y^{(n)})=f_n(x)y^{(n)}+\cdots+f_1(x)y'+f_0(x)+Q(x)=0,
            \end{equation*}
            where all \(f_i=a_i\), \(0\leq i\leq n\), where \(a_i\in\mathbb{C}\) and \(Q\) are continuous on some common interval \(I\).
            
        \end{definition}
        \begin{definition}{\Stop\,\,Homogeneous Linear Differential Equations}{homolindiffeq}

            An \(n\)th order linear homogeneous differential equation, with respect to \(I\), is 
            \begin{equation*}
                F(x,y,y',\ldots,y^{(n)})=f_n(x)y^{(n)}+\cdots+f_1(x)y'+f_0(x)+Q(x)=0,
            \end{equation*}
            where all \(f_i\), \(0\leq i\leq n\), and \(Q\) are continuous on some common interval \(I\) and \(f_n(x)\neq0\) and \(Q(x)=0\) for all \(x\in I\). 
            
        \end{definition}
        \vphantom
        \\
        \\
        Consider the following theorem.
        \begin{theorem}{\Stop\,\,Decomposing a Solution of a Linear Differential Equation}{decompsol}

            Let \(F(x,y,y',\ldots,y^{(n)})=f_n(x)y^{(n)}+\cdots+f_1(x)y'+f_0(x)=0\) have the solution
            \begin{equation*}
                y_c(x,c_1,\ldots,c_n)=c_1y_1(x)+\cdots+c_ny_n(x).
            \end{equation*}
            Then, the solution to
            \begin{equation*}
                \tilde{F}(x,y,y',\ldots,y^{(n)})=f_n(x)y^{(n)}+\cdots+f_1(x)y'+f_0(x)+Q(x)=0
            \end{equation*}
            is given by
            \begin{equation*}
                y(x,c_1,\ldots,c_n)=y_c(x,c_1,\ldots,c_n)+y_p(x)
            \end{equation*}
            where \(y_p(x)\) is any particular solution to \(\tilde{F}\).

        \end{theorem}

\pagebreak

\section{Lecture 14: February 24, 2023}

    \subsection{Solving Linear Homogeneous Equations With Constant Coefficients: Part I}

        Consider the second order homogeneous linear differential equation with constant coefficients
        \begin{equation*}
            F(x,y,y',y'')=ay''+by'+cy=0,
        \end{equation*}
        for \(a,b,c\in\mathbb{C}\). We may, without loss of generality, rewrite this as
        \begin{equation*}
            \tilde{F}(x,y,y',y'')=y''+Ay'+By=0,
        \end{equation*}
        since \(a\neq0\). Choose some \(\alpha,\beta\in\mathbb{C}\) such that \(-(\alpha+\beta)=A\) and \(\alpha\beta=B\). Now, consider \(z=y'(x)-\alpha y(x)\). Then, note that \(\frac{\dd z}{\dd x}-\beta z=0\). That is,
        \begin{align*}
            0&=y''-\alpha y'(x)-\beta y'(x)+\alpha\beta y(x) \\
            &=y''-(\alpha+\beta)y'+2\beta y \\
            &=y''+Ay'+By.
        \end{align*}
        Now, since \(\frac{\dd z}{\dd x}-\beta z=0\), we have, by separation of variables, \(z=c_1e^{\beta x}\) where \(c_1\in\mathbb{C}\) and \(x\in\mathbb{R}\).  Now, we have
        \begin{equation*}
            z=y'(x)-\alpha y(x)=c_1e^{\beta x}.
        \end{equation*}
        We may use the integrating factor \(\mu(x)=e^{\int -\alpha \dd x}=e^{-\alpha x}\). Consider
        \begin{align*}
            y'(x)e^{-\alpha x}-\alpha y(x)e^{-\alpha x}&=c_1e^{\beta x}e^{-\alpha x} \\
            &=\frac{\dd}{\dd x}(y(x)e^{-\alpha x}).
        \end{align*}
        By integration,
        \begin{align*}
            y(x)&=e^{\alpha x}\int c_1e^{(\beta-\alpha) x} \dd x \\
            &=
            \begin{cases}
                e^{\alpha x}(c_1x+c_2)=c_1xe^{\alpha x}+c_2e^{\alpha x} & \alpha=\beta \\
                e^{\alpha x}\left(\frac{c_1e^{(\beta-\alpha)x}}{\beta-\alpha}+c_2\right)=\frac{c_1}{\beta-\alpha}e^{\beta x}+c_2e^{\alpha x}=\tilde{c}_1e^{\beta x}+c_2e^{\alpha x} & \alpha\neq\beta
            \end{cases}.
        \end{align*}
        In the first case \(\alpha=\beta\), we have that \(\alpha=\beta\) is a double root of the polynomial \(m^2-2\alpha m+\alpha^2=0\). In the second case, \(\alpha\) and \(\beta\) are distinct solutions to the polynomial \(m^2-(\alpha+\beta)m+\alpha\beta=0\). Both solutions are truly general solutions, because there are \(n=2\) linearly independent terms in the linear combination.  
        \pagebreak

\section{Lecture 15: February 28, 2023}

    \subsection{Solving Linear Homogeneous Equations With Constant Coefficients: Part II}

        With the discoveries in the previous section, we provide the following definition and theorem.
        \begin{definition}{\Stop\,\,Characteristic Polynomials}{characteristicpolynomial}

            Let \(F(x,y,\ldots,y^{(n)})_{\text{hom}}=a_ny^{(n)}+\cdots+a_0y=0\) be an ordinary differential equation. The characteristic polynomial of \(F\) is given by
            \begin{equation*}
                p_F(m)=a_nm^n+\cdots+a_0m^0.
            \end{equation*}
            
        \end{definition}
        \begin{theorem}{\Stop\,\,Finding Solutions With Characteristic Polynomials}{usingcharpoly}

            If \(F(x,y,\ldots,y^{(n)})_{\text{hom}}=a_ny^{(n)}+\cdots+a_0y=0\) has a characteristic polynomial with factors into distinct factors,
            \begin{equation*}
                y(x,c_1,\ldots c_n)=c_1e^{m_1x}+\cdots+c_ne^{m_nx},
            \end{equation*}
            If \(F(x,y,\ldots,y^{(n)})_{\text{hom}}=a_ny^{(n)}+\cdots+a_0y=0\) has a characteristic polynomial that factors into \(k\) distinct roots, \(k<n\), that is,
            \begin{equation*}
                p_F(m)=(m-m_1)^{r_1}+\cdots+(m-m_k)^{r_k},
            \end{equation*}
            where \(r_1+\cdots+r_k=n\). We have
            \begin{equation*}
                y(x,c_1,\ldots,c_n)=(c_1x^0+\cdots+c_{r_1}x^{r_1-1})e^{m_1x}+\cdots+(c_{n-r_k}x^0+\cdots+c_nx^{n-r_k})e^{m_nx}.
            \end{equation*}
            
        \end{theorem}
        \vphantom
        \\
        \\
        We will now provide an important remark about Theorem \ref{thm:usingcharpoly} for the ease of understanding. Let \(p_F(m)\) be a characteristic polynomial with root \(m_k\in\mathbb{C}\). We may use Euler's Formula to remove imaginary exponents from the \(n\)th parameter family given by Theorem \ref{thm:usingcharpoly}. For convenience, a statement of Euler's Formula is given in Theorem \ref{thm:eulerformula}.
        \pagebreak
        \begin{theorem}{\Stop\,\,Euler's Formula}{eulerformula}

            For \(\theta\in\mathbb{R}\),
            \begin{equation*}
                e^{i\theta}=\cos\theta+i\sin\theta.
            \end{equation*}
            \begin{proof}
                Recall that \(i^0=i^4=i^8=\cdots=1\), \(i^1=i^5=i^9=\cdots=i\), \(i^2=i^6=i^10=\cdots=-1\), and \(i^3=i^7=i^{11}=\cdots=-i\). If we define, for \(z\in\mathbb{C}\), 
                \begin{equation*}
                    e^z=\sum_{k=0}^\infty \frac{z^k}{k!},
                \end{equation*}
                we have that
                \begin{align*}
                    e^{i\theta}&=1+ix+\frac{(ix)^2}{2!}+\frac{(ix)^3}{3!}+\frac{(ix)^4}{4!}+\frac{(ix)^5}{5!}+\cdots \\
                    &=1+ix-\frac{x^2}{2!}-\frac{ix^3}{3!}+\frac{x^4}{4!}+\frac{ix^5}{5!}+\cdots \\
                    &=\left(1-\frac{x^2}{2!}+\frac{x^4}{4!}+\cdots\right)+i\left(x-\frac{x^3}{3!}+\frac{x^5}{5!}+\cdots\right) \\
                    &=\cos\theta+i\sin\theta,
                \end{align*}
                as desired.
            \end{proof}
            
        \end{theorem}
        \vphantom
        \\
        \\
        To use Euler's Formula, suppose that \(p_L(m)\) is our characteristic polynomial with root \(m_k=a+bi\in\mathbb{C}\). Then, \(\bar{m_\ell}=a-bi\in\mathbb{C}\) is also a root of \(p_L(m)\). Without loss of generality, suppose both \(m_k\) and \(m_\ell\) have multiplicity \(1\). Let our solution components be
        \begin{equation*}
            y_k(x)=e^{(a+bi)x}=e^{ax}(\cos(bx)+i\sin(bx)),\quad y_\ell(x)=e^{(a-bi)x}=e^{ax}(\cos(bx)-i\sin(bx)).
        \end{equation*}
        Since the linear combination of \(y_k\) and \(y_\ell\) is a solution, \(\frac{1}{2}y_k(x)+\frac{1}{2}y_\ell(x)=e^{ax}\cos(bx)\) is a solution, and \(\frac{1}{2i}y_k(x)-\frac{1}{2i}y_\ell(x)=e^{ax}\sin(bx)\) is also a solution. Thus,
        \begin{equation*}
            y_{k,\ell}=c_ke^{ax}\cos(bx)+c_\ell e^{ax}\sin(bx)
        \end{equation*}
        is a solution.
        \pagebreak
        \\
        \\
        Consider the following examples.
        \begin{example}{\Difficulty\,\Difficulty\,\,Real Roots 1}{realroots1}
            
            Find a general solution to
            \begin{equation*}
                2y''+y'-6y=0.
            \end{equation*}
            The characteristic polynomial is
            \begin{equation*}
                2m^2+m-6=0,
            \end{equation*}
            which can be rewritten as
            \begin{equation*}
                (2m-3)(m+2)=0.
            \end{equation*}
            We can see that \(m=-2\) and \(m=\frac{3}{2}\) are solutions to the characteristic polynomial. Therefore,
            \begin{equation*}
                y=c_1e^{-2x}+c_2e^{\frac{3}{2}x}.
            \end{equation*}
        
        \end{example}
        \begin{example}{\Difficulty\,\Difficulty\,\,Real Roots 2}{realroots2}
        
            Find a general solution to
            \begin{equation*}
                y'''-6y''+11y'-6y=0.
            \end{equation*}
            The characteristic polynomial is
            \begin{equation*}
                m^3-6m^2+11m-6=0.
            \end{equation*}
            We may carry out polynomial long division, with the substitution \(x=m\), as follows
            \begin{equation*}
                \polylongdiv{x^3-6x^2+11x-6}{x-1}.
            \end{equation*}
            We can see that \(m=1\), \(m=2\), and \(m=3\) are solutions to the characteristic polynomial. Therefore, 
            \begin{equation*}
                y=c_1e^x+c_2e^{2x}+c_3e^{3x}.
            \end{equation*}
        
        \end{example}
        \pagebreak
        \begin{example}{\Difficulty\,\Difficulty\,\,Real Roots 3}{realroots3}
        
            Find a general solution to
            \begin{equation*}
                2y'''-7y''+4y'+4y=0.
            \end{equation*}
            The characteristic polynomial is
            \begin{equation*}
                2m^3-7m^2+4m+4=0.
            \end{equation*}
            We may carry out polynomial long division, with the substitution \(x=m\), as follows
            \begin{equation*}
                \polylongdiv{2x^3-7x^2+4x+4}{x-2}.
            \end{equation*}
            Therefore, the characteristic polynomial may be rewritten as
            \begin{equation*}
                (m-2)(2m^2-3m-2)=0.
            \end{equation*}
            The above quadratic can be factored as
            \begin{equation*}
                (2m+1)(m-2).
            \end{equation*}
            Therefore, the characteristic polynomial is
            \begin{equation*}
                (m-2)^2(2m+1)=0.
            \end{equation*}
            Therefore,
            \begin{equation*}
                y=c_1e^{2x}+c_2xe^{2x}+c_3e^{-\frac{1}{2}x}.
            \end{equation*}
        
        \end{example}
        \pagebreak
        \begin{example}{\Difficulty\,\Difficulty\,\,Real Roots 4}{realroots4}
        
            Find a general solution to
            \begin{equation*}
                y''''-y'''-13y''+y'+12y=0.
            \end{equation*}
            The characteristic polynomial is
            \begin{equation*}
                m^4-m^3-13m^2+m+12=0.
            \end{equation*}
            We may carry out polynomial long division, with the substitution \(x=m\), as follows
            \begin{equation*}
                \polylongdiv{x^4-x^3-13x^2+x+12}{x-1}.
            \end{equation*}
            Therefore, the characteristic polynomial may be rewritten as
            \begin{equation*}
                (m-1)(m^3-13m-12)=0.
            \end{equation*}
            We may also perform polynomial long division with the above cubic, with the same substitution as before, producing
            \begin{equation*}
                \polylongdiv{x^3-13x-12}{x+1}.
            \end{equation*}
            Therefore, we have
            \begin{equation*}
                (m-1)(m+1)(m-4)(m+3)=0.
            \end{equation*}
            Therefore,
            \begin{equation*}
                y=c_1e^{-x}+c_2e^{x}+c_3e^{4x}+c_4e^{-3x}.
            \end{equation*}
            
        \end{example}
        \pagebreak
        \begin{example}{\Difficulty\,\Difficulty\,\,Real Roots 5}{realroots5}
        
            Find a general solution to
            \begin{equation*}
                y'''''-4y''''-8y'''+14y''+7y'-10y=0.
            \end{equation*}
            The characteristic polynomial is
            \begin{equation*}
                m^5-4m^4-8m^3+14m^2+7m-10=0.
            \end{equation*}
            We may carry out polynomial long division, with the substitution \(x=m\), as follows
            \begin{equation*}
                \polylongdiv{x^5-4x^4-8x^3+14x^2+7x-10}{x-1}.
            \end{equation*}
            Therefore, the characteristic polynomial may be rewritten as
            \begin{equation*}
                (m-1)(m^4-3m^3-11m^2+3m+10)=0.
            \end{equation*}
            We may also perform polynomial long division with the above quartic, with the same substitution as before, producing
            \begin{equation*}
                \polylongdiv{x^4-3x^3-11x^2+3x+10}{x+2}.
            \end{equation*}
            We then obtain
            \begin{equation*}
                (m-1)(m+2)(m^3-5m^2-m+5)=0.
            \end{equation*}
            We may proceed similarly to produce
            \begin{equation*}
                (m-1)^2(m+2)(m-5)(m+1)=0.
            \end{equation*}
            Therefore,
            \begin{equation*}
                y=c_1e^x+c_2xe^x+c_3e^{-2x}+c_4e^{-x}+c_5e^{5x}.
            \end{equation*}
            
        \end{example}
        \begin{exercise}{\Difficulty\,\Difficulty\,\,Complex Roots 1}{complexroots1}
        
            Find a general solution to
            \begin{equation*}
                9y''+6y'+4y=0.
            \end{equation*}
            The characteristic polynomial is
            \begin{equation*}
                9m^2+6m+4.
            \end{equation*}
            By the quadratic formula,
            \begin{equation*}
                m=-\frac{1}{3}\pm\frac{\sqrt{3}}{3}.
            \end{equation*}
            Therefore,
            \begin{equation*}
                y=e^{-\frac{1}{3}}\left(c_1\cos\left(x\frac{\sqrt{3}}{3}\right)+c_2\sin\left(x\frac{\sqrt{3}}{3}\right)\right).
            \end{equation*}
            
        \end{exercise}
        \begin{exercise}{\Difficulty\,\Difficulty\,\,Complex Roots 2}{complexroots2}
        
            Find a general solution to
            \begin{equation*}
                y'''+27y'=0.
            \end{equation*}
            The characteristic polynomial is
            \begin{equation*}
                m^3+27m=0,
            \end{equation*}
            or
            \begin{equation*}
                m(m^2+27)=0,
            \end{equation*}
            which can further be factored as
            \begin{equation*}
                m(m+3i\sqrt{3})(m-3i\sqrt{3})=0.
            \end{equation*}
            Solutions are then \(m=0\) and \(m=\pm3i\sqrt{3}\). Therefore,
            \begin{equation*}
                y=c_1+c_2\cos(3x\sqrt{3})+c_3\sin(3x\sqrt{3}).
            \end{equation*}
        
        \end{exercise}